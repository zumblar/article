Построим выделенное формальное решение для полной системы, которое будем называть \textit{каноническим} и обозначать верхним индексом $0$.

Система уравнений на первый порядок является однородной, поэтому для рассматриваемого случая выберем тривиальное решение:
\begin{align*}
    \v u^0_1(\tau) \equiv \begin{pmatrix} x^0_1(\tau) \\ y^0_1(\tau) \end{pmatrix} = \begin{pmatrix} 0 \\ 0 \end{pmatrix}.
\end{align*}

В силу этого $(\lambda^0_1(\tau), \Lambda^0_1(\tau)) = (0,0)$. В таком случае:
$$\mathfrak{D}_1 = 0 \cdot \quad \text{ - оператор умножения на 0},$$
$$G_2^\rho = \frac12 \mathfrak{D}_1^2 g^\rho = \frac12 (x^0_1)^2 g^\rho_{xx} + \frac12 x^0_1 y^0_1 g^\rho_{xy} + ... = 0,$$

$$\tilde G_2^x  = \frac{1}{\alpha \beta} \lambda_0''(\tau) \left( -2Cy_0 \sin\lambda_0 - (2Cx_0+2C \hat x_0+e_JD) \cos\lambda_0 \right) \text{ - нечетная},$$

$$\tilde G_2^y  = \frac{1}{\alpha \beta} \lambda_0''(\tau) \left( -(2Cx_0+2C \hat x_0+e_JD) \sin\lambda_0 + 2Cy_0 \cos\lambda_0 \right) \text{ - четная}.$$

При построении $\v u_2^u,\v u_2^s$ выберем $\tau^u_2 = \tau^s_2 = 0$:
\begin{equation*}
    \begin{cases}
        \v u_2^u = 
        \v u_1(\tau)\bigint_{\text{ } 0}^\tau \left( \tilde y_1(\tau) \tilde G_2^x(\tau) - \tilde x_1(\tau) \tilde G_2^y(\tau) \right) d \tau + 
        \v{ \tilde{u_1}}(\tau) \bigint_{-\infty}^\tau \left( x_1(\tau) \tilde G_2^y(\tau) - y_1(\tau) \tilde G_2^x(\tau) \right) d \tau,\\
        \\
        
        \v u_2^s = 
        \v u_1(\tau)\bigint_{\text{ } 0}^{\tau} \left( \tilde y_1(\tau) \tilde G_2^x(\tau) - \tilde x_1(\tau) \tilde G_2^y(\tau) \right) d \tau - 
        \v{ \tilde{u_1}}(\tau) \bigint_{\text{ } \tau}^{+\infty} \left( x_1(\tau) \tilde G_2^y(\tau) - y_1(\tau) \tilde G_2^x(\tau) \right) d \tau. \\
    \end{cases}
\end{equation*}

Тогда по \textit{Следствию 1} $\v u_2^s(\tau) = \v u_2^u(\tau) \equiv \v u_2^0(\tau)$

Т.к. $G_k^\rho$ и $\hat G_k^\rho$ отличаются только нормировкой то вырождаются они одновременно. В таком случае:

$$G_3^\rho = \frac12 (\mathfrak{D}_1 \mathfrak{D}_2 + \mathfrak{D}_2 \mathfrak{D}_1)g^\rho + \mathfrak{D}_1 \hat G_2^\rho = 0,$$
\begin{equation*}
\begin{aligned}
\tilde G_3^x  = \frac{1}{\alpha \beta} \underbrace{(\lambda^0_1)''(\tau)}_0 \left( -2Cy_0 \sin\lambda_0 - (2Cx_0+2C \hat x_0+e_JD) \cos\lambda_0 \right) =0,
\end{aligned}
\end{equation*}
\begin{equation*}
\begin{aligned}
\tilde G_3^y  = \frac{1}{\alpha \beta} \underbrace{(\lambda^0_1)''(\tau)}_0 \left( -(2Cx_0+2C \hat x_0+e_JD) \sin\lambda_0 + 2Cy_0 \cos\lambda_0 \right) = 0.
\end{aligned}
\end{equation*}

То есть в 3 порядке неоднородность вырождается и уравнение становится однородным. Выберем тривиальное решение:
\begin{align*}
    \v u^0_3(\tau) \equiv \begin{pmatrix} x^0_3(\tau) \\ y^0_3(\tau) \end{pmatrix} = \begin{pmatrix} 0 \\ 0 \end{pmatrix},
\end{align*}

$$(\lambda^0_3(\tau), \Lambda^0_3(\tau)) = (0,0).$$

Продолжая данное построение по индукции получаем следующее

%%%%%%%%%%%%%%%%%%%%%%%%%%%%%%%%%%%%%%%%%%%%%%%%%%%%%%%%%%%%%%%%%%%%%%%%%%%%%%%%%%%%%%%%%%%%%%%%%%%%%%%%%%%%%%%%%%%%%%%%%%%%%%%%%%%%%
\begin{utv} Для любого целого $k\ge 0$ имеем

$$G_{2k+1}^\rho = 0, \rho \in \{\Lambda,\lambda,x,y\},$$
$$G_{2k+2}^x, y_{2k+2} \text{ -- нечетные},$$
$$G_{2k+2}^y, x_{2k+2} \text{ -- четные},$$
$$({\lambda^0_{2k+1}, \Lambda^0_{2k+1}, x^0_{2k+1}, y^0_{2k+1}}) = (0,0,0,0).$$

\end{utv}
%%%%%%%%%%%%%%%%%%%%%%%%%%%%%%%%%%%%%%%%%%%%%%%%%%%%%%%%%%%%%%%%%%%%%%%%%%%%%%%%%%%%%%%%%%%%%%%%%%%%%%%%%%%%%%%%%%%%%%%%%%%%%%%%%%%%%
\textbf{Доказательство:}\nopagebreak[4]

База индукции описана ранее.
Индукционное предположение:
$$G_{2k-1}^\rho = 0,$$
$$G_{2k}^x, y_{2k} \text{ -- нечетные},$$
$$G_{2k}^y, x_{2k} \text{ -- четные},$$
$$({\lambda^0_{2k-1}, \Lambda^0_{2k-1}, x^0_{2k-1}, y^0_{2k-1}}) = (0,0,0,0).$$

По \textit{Утверждению 5} 
\begin{equation}
G_{2k+1}^\rho = \frac12 \sum_{m=1}^{2k} \mathfrak{D}_m \mathfrak{D}_{2k+1-m} g^\rho + \sum_{m=1}^{2k} \mathfrak{D}_m \hat G_{2k+1-m}^\rho.
\label{G_nechet}
\end{equation}

В силу индукционного предположения, для всех нечетных $j \leq 2k-1$ операторы $\mathfrak{D}_{j}$ это операторы умножения на $0$.

Первая сумма в (\ref{G_nechet}) состоит из произведений $\mathfrak{D}_m \mathfrak{D}_{2k+1-m} g^\rho$, сумма индексов которых должна быть нечетна и равна $2k+1$. Следовательно, либо $\mathfrak{D}_m$, либо $\mathfrak{D}_{2k+1-m}$ имеет нечетный индекс.  
Тогда $\mathfrak{D}_m \mathfrak{D}_{2k+1-m} g^\rho = 0 \quad \forall m \leq 2k$.

Аналогично во второй сумме в (\ref{G_nechet}) нечетный индекс имеет либо $\mathfrak{D}_m$, либо $\hat G^\rho_{2k+1-m}$. Но поскольку для нечетных $j \le 2k-1$ функции $\hat G^\rho_{j} \equiv 0$, то все слагаемые во второй сумме также обращаются в 0. 

Также в силу индукционного предположения и (\ref{G_nechet}) неоднородность в нечетных уравнениях также будет нулевой:
\begin{equation*}
\begin{aligned}
\tilde G_{2k+1}^x  = \frac{1}{\alpha \beta} \underbrace{(\lambda^0_{2k-1})''(\tau)}_0 \left( -2Cy_0 \sin\lambda_0 - (2Cx_0+2C \hat x_0+e_JD) \cos\lambda_0 \right) =0,
\end{aligned}
\end{equation*}
\begin{equation*}
\begin{aligned}
\tilde G_{2k+1}^y  = \frac{1}{\alpha \beta} \underbrace{(\lambda^0_{2k-1})''(\tau)}_0 \left( -(2Cx_0+2C \hat x_0+e_JD) \sin\lambda_0 + 2Cy_0 \cos\lambda_0 \right) = 0.
\end{aligned}
\end{equation*}

Тогда можно выбрать:
$$({\lambda^0_{2k+1}, \Lambda^0_{2k+1}, x^0_{2k+1}, y^0_{2k+1}}) = (0,0,0,0).$$

Заметим, что оператор $\mathfrak{D}_{2k}$ сохраняет четность:

\begin{equation*}
\begin{aligned}
\mathfrak{D}_{2k}g^\rho = \underbrace{x^0_{2k}}_{\text{чет.}} g^\rho_x|_{\lambda_0,\Lambda_0,x_0,y_0} + \underbrace{y^0_{2k}}_{\text{нечет.}} g^\rho_y|_{\lambda_0,\Lambda_0,x_0,y_0} + \underbrace{\lambda^0_{2k}}_{\text{нечет.}} g^\rho_\lambda|_{\lambda_0,\Lambda_0,x_0,y_0} + \underbrace{\Lambda^0_{2k}}_{(\lambda^0_{2k-1})' = 0} g^\rho_\Lambda|_{\lambda_0,\Lambda_0,x_0,y_0}
\end{aligned}
\end{equation*}

$\frac{\partial}{\partial x}|_{\lambda_0,\Lambda_0,x_0,y_0}$ не меняет четность, $\frac{\partial}{\partial y}|_{\lambda_0,\Lambda_0,x_0,y_0}, \frac{\partial}{\partial \lambda}|_{\lambda_0,\Lambda_0,x_0,y_0}$ меняют четность.


Тогда, как следствие \textit{Утверждения 5}, $G_{2k}^\rho$ наследует четность $G_{2k-2}^\rho$.
$\blacksquare$
%%%%%%%%%%%%%%%%%%%%%%%%%%%%%%%%

В итоге построенное \textit{каноническое} решение имеет вид:

\begin{equation*}
\begin{cases}
\lambda^0(\tau) = (\lambda_0(\tau) - \pi) + \sum_{k=1}^{+\infty} \varepsilon^{2k} \lambda^0_{2k}(\tau),\\
\Lambda^0(\tau) = \frac{1}{\alpha} \sum_{k=1}^{+\infty} \varepsilon^{2k} (\lambda^0_{2k-1})'(\tau),\\
x^0(\tau) = x_0(\tau) + \sum_{k=1}^{+\infty} \varepsilon^{2k} x^0_{2k}(\tau),\\
y^0(\tau) = y_0(\tau) + \sum_{k=1}^{+\infty} \varepsilon^{2k} y^0_{2k}(\tau),
\end{cases}
\end{equation*}

\begin{equation*}
\begin{cases}
        x^0_{2k}(\tau) = 
        x_1(\tau)\int_{\text{ } 0}^\tau \left( \tilde y_1(\tau) \tilde G_{2k}^x(\tau) - \tilde x_1(\tau) \tilde G_{2k}^y(\tau) \right) d \tau + 
        \tilde {x_1}(\tau) \int_{-\infty}^\tau \left( x_1(\tau) \tilde G_{2k}^y(\tau) - y_1(\tau) \tilde G_{2k}^x(\tau) \right) d \tau, \\
        
        y^0_{2k}(\tau) = 
        y_1(\tau)\int_{\text{ } 0}^\tau \left( \tilde y_1(\tau) \tilde G_{2k}^x(\tau) - \tilde x_1(\tau) \tilde G_{2k}^y(\tau) \right) d \tau + 
        \tilde {y_1}(\tau) \int_{-\infty}^\tau \left( x_1(\tau) \tilde G_{2k}^y(\tau) - y_1(\tau) \tilde G_{2k}^x(\tau) \right) d \tau,
\end{cases}
\end{equation*}

$$\tilde G_{2k}^x  = \frac{1}{\alpha \beta} (\lambda^0_{2k-2})''(\tau) \left( -2Cy_0 \sin\lambda_0 - (2Cx_0+2C \hat x_0+e_JD) \cos\lambda_0 \right),$$

$$\tilde G_{2k}^y  = \frac{1}{\alpha \beta} (\lambda^0_{2k-2})''(\tau) \left( -(2Cx_0+2C \hat x_0+e_JD) \sin\lambda_0 + 2Cy_0 \cos\lambda_0 \right).$$
%%%%%%%%%%%%%%%%%%%%%%%%%%%%%%%%%%%%%%%%%%%%%%%%%%%%%%%%%%%%%%%%%%%%%%%%%%%%%%%%%%%%%%%%%%%%%%%%%%%%%%%%%%%%%%%%%%%%%%%%%%%%%%%%%%%%%
\begin{thm}
\begin{samepage}
Существуют решения системы (\ref{fulltn}):
\begin{itemize}
\item $(\Lambda^s(\tau),\lambda^s(\tau),x^s(\tau),y^s(\tau)) \in W^s(0),$
\item $(\Lambda^u(\tau),\lambda^u(\tau),x^u(\tau),y^u(\tau)) \in W^u(0),$
\end{itemize}

асимптотика которых совпадает в любом порядке по $\varepsilon$ и равна $(\Lambda^0(\tau),\lambda^0(\tau),x^0(\tau),y^0(\tau))$.
\end{samepage}
То есть $\exists \quad c>0:$
$$(\Lambda^s(\tau),\lambda^s(\tau),x^s(\tau),y^s(\tau)) = (\Lambda^0(\tau),\lambda^0(\tau),x^0(\tau),y^0(\tau)) + \mathcal{O}(e^{-\frac{c}{\varepsilon}}),$$
$$(\Lambda^u(\tau),\lambda^u(\tau),x^u(\tau),y^u(\tau)) = (\Lambda^0(\tau),\lambda^0(\tau),x^0(\tau),y^0(\tau)) + \mathcal{O}(e^{-\frac{c}{\varepsilon}}).$$

\end{thm}
%%%%%%%%%%%%%%%%%%%%%%%%%%%%%%%%%%%%%%%%%%%%%%%%%%%%%%%%%%%%%%%%%%%%%%%%%%%%%%%%%%%%%%%%%%%%%%%%%%%%%%%%%%%%%%%%%%%%%%%%%%%%%%%%%%%%%
\textbf{Доказательство:}\nopagebreak[4]

Функции 
$$\int_{\text{ } 0}^\tau \left( \tilde y_1(\tau) \tilde G_{2k}^x(\tau) - \tilde x_1(\tau) \tilde G_{2k}^y(\tau) \right) d \tau,$$
равномерно ограничены $\forall k \in \mathbb{N}$, т.к. подынтегральное выражение принадлежит классу $\mathcal{F}_{s_0}$.

В силу нечетности подынтегрального выражения (по \textit{Утверждению 7}) следущие интегралы равны 0:
$$\int_{-\infty}^{+\infty} \left( x_1(\tau) \tilde G_{2k}^y(\tau) - y_1(\tau) \tilde G_{2k}^x(\tau) \right) d \tau, \quad k \in \mathbb{N}$$

Тогда, учитывая что подынтегральное выражение принадлежит классу $\mathcal{F}_{3s_0}$, функции
$$\int_{-\infty}^{\tau} \left( x_1(\tau) \tilde G_{2k}^y(\tau) - y_1(\tau) \tilde G_{2k}^x(\tau) \right) d \tau$$
будут принадлежать классу $\mathcal{F}_{2s_0}$. Причем их убывание будет подавлять рост $\v{ \tilde{u_1}}$.

Таким образом, $(\Lambda^0(\tau),\lambda^0(\tau),x^0(\tau),y^0(\tau))$ стремится к нулю при $\tau \to \pm \infty$.

Заметим, что правая часть $(g^\Lambda, g^\lambda, g^x, g^y)$ системы (\ref{fulltn}) является вещественно аналитической по $(\Lambda, \lambda, x, y)$ и $\varepsilon>0$. Также отметим, что все производные $(g^\Lambda, g^\lambda, g^x, g^y)$ по $(\Lambda, \lambda, x, y)$ равномерно ограничены. Такие свойства правой части гарантируют вещественную аналитичность решений.

Выберем два решения, принадлежащих $W^s(0)$ и $W^u(0)$ соответсвенно и имеющих одинаковую асимптотику $(\Lambda^0(\tau),\lambda^0(\tau),x^0(\tau),y^0(\tau))$. Тогда по лемме Ватсона их разность будет экспоненциально мала.


$\blacksquare$
%%%%%%%%%%%%%%%%%%%%%%%%%%%%%%%%%%%%%%%%%%%%%%%%%%%%%%%%%%%%%%%%%%%%%%%%%%%%%%%%%%%%%%%%%%%%%%%%%%%%%%%%%%%%%%%%%%%%%%%%%%%%%%%%%%%%%
\begin{consequence}
Расщепление сепаратрис положения равновесия седло-центр в системе (\ref{fulltn}) не более чем экспоненциально мало. 
\end{consequence}
%%%%%%%%%%%%%%%%%%%%%%%%%%%%%%%%%%%%%%%%%%%%%%%%%%%%%%%%%%%%%%%%%%%%%%%%%%%%%%%%%%%%%%%%%%%%%%%%%%%%%%%%%%%%%%%%%%%%%%%%%%%%%%%%%%%%%
