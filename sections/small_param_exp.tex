В данном разделе построим формальные решения системы (\ref{fulltn}), параметризующие устойчивое и неустойчивое многообразия $W^{s,u}(0)$ неподвижной точки. Для этого рассмотрим в дополнение к уравнениям (\ref{fulltn}) следующие начальные условия:
    
\begin{itemize}
\item Решения принадлежащие $W^s(0)$:
\begin{equation*}
    \begin{cases}
        \Lambda^{s}(+ \infty) = 0, \\
        \lambda^{s}(+ \infty) = 0,\\
        x^{s}(+ \infty) = 0, \\
        y^{s}(+ \infty) = 0,
    \end{cases}
    \label{border}
\end{equation*}
\item Решения принадлежащие $W^u(0)$:
\begin{equation*}
    \begin{cases}
        \Lambda^{u}(- \infty) = 0, \\
        \lambda^{u}(- \infty) = 0,\\
        x^{u}(- \infty) = 0, \\
        y^{u}(- \infty) = 0.
    \end{cases}
\end{equation*}
\end{itemize}

%    Задачи Коши (\ref{fullt}), (\ref{border}) эквивалентны следующим интегральным уравнениям
%\begin{equation}
%    \begin{cases}
%        \Lambda(t) = \int\limits_{\pm\infty}^t u \big(x(t'),y(t') \big) \sin \lambda(t') - v \big(x(t'),y(t') \big) \cos \lambda(t') dt', \\
%        \lambda(t) = \lambda(\hat x_0,0) + \int\limits_{\pm\infty}^t \alpha \Lambda(t') dt', \\
%        x(t) = \hat x_0 - \varepsilon \int\limits_{\pm\infty}^t \Big( 2Fy(t')+\frac{\partial u}{\partial y} \cos \lambda(t') + \frac{\partial v}{\partial y} \sin \lambda(t') \Big) dt', \\
%        y(t) = \varepsilon \int\limits_{\pm\infty}^t \Big( 2F \big(x(t')+ \hat x_0 \big)+e_JG +\frac{\partial u}{\partial x} \cos \lambda(t') + \frac{\partial v}{\partial x} \sin \lambda(t') \Big)dt'. \\
%    \end{cases}
%    \label{fulltint}
%\end{equation}
   

    Представим формальные решения системы (\ref{fulltn}) в виде рядов:
    \begin{equation}
    \begin{cases}
\Lambda^{s,u}(\tau) = \sum_{k=0}^\infty \varepsilon^k \Lambda^{s,u}_k(\tau)\\
\lambda^{s,u}(\tau) = \sum_{k=0}^\infty \varepsilon^k \lambda^{s,u}_k(\tau)\\
x^{s,u}(\tau) =       \sum_{k=0}^\infty \varepsilon^k x^{s,u}_k(\tau)\\
y^{s,u}(\tau) =       \sum_{k=0}^\infty \varepsilon^k y^{s,u}_k(\tau)\\
\end{cases}
    \label{predst}
\end{equation}


    и подставим их в уравнения, перейдя к медленному времени $\tau=\varepsilon t$.
    Приравнивая коэффициенты при одинаковых степенях $\varepsilon$, получаем набор уравнений, причем в силу того, что система содержит 2 временных масштаба, уравнения для $(\lambda_k, \Lambda_k), k \ge 1$ будут алгебраическими:

\begin{equation}
    \begin{cases}
        \Lambda_k(\tau) = \frac{1}{\alpha} \lambda_{k-1}', \\
        
        \lambda_k(\tau) = \frac1\beta \Big( \alpha^{-1} \lambda_{k-2}'' - G_k^\Lambda + x_k \big( (2Cx_0+2C \hat x_0+e_JD) \sin\lambda_0 - 2Cy_0 \cos\lambda_0 \big) + \\
        
        \quad \quad \quad +y_k \big( -2Cy_0 \sin\lambda_0 - (2Cx_0+2C \hat x_0+e_JD) \cos\lambda_0 \big) \Big),\\
        \lambda_{-1}=0,
    \end{cases}
\end{equation}

а для $(x_k,y_k), k \ge 1$ - дифференциальными:
\begin{equation}
\frac{d}{d\tau} \begin{pmatrix} x_k \\ y_k \end{pmatrix} = \begin{pmatrix} a(\tau) & b(\tau)\\ c(\tau) & -a(\tau) \end{pmatrix} \begin{pmatrix} x_k \\ y_k \end{pmatrix} + \begin{pmatrix} \tilde G_k^x(\tau) \\ \tilde G_k^y(\tau) \end{pmatrix} \equiv \mathcal{A}(\tau) \begin{pmatrix} x_k \\ y_k \end{pmatrix} + \begin{pmatrix} \tilde G_k^x(\tau) \\ \tilde G_k^y(\tau) \end{pmatrix},
\label{xy_eq}
\end{equation}
где:
$$ \tilde G_k^x = G_k^x - \beta^{-1} (\alpha^{-1} \lambda_{k-2}'' - G_k^\Lambda) \big( -2Cy_0 \sin\lambda_0 - (2Cx_0+2C \hat x_0+e_JD) \cos\lambda_0 \big), $$
$$ \tilde G_k^y = G_k^y - \beta^{-1} (\alpha^{-1} \lambda_{k-2}'' - G_k^\Lambda) \big( -(2Cx_0+2C \hat x_0+e_JD) \sin\lambda_0 + 2Cy_0 \cos\lambda_0 \big). $$


Здесь $a,b,c$ - функции зависящие от $x_0(\tau), y_0(\tau), \lambda_0(\tau), \Lambda_0(\tau)$:
$$a(\tau) = \frac{\left( e_J^2 D^2 - 4CEe_J^2\right) \sin \lambda_0 \cos \lambda_0}{\beta(x_0,y_0)},$$
$$b(\tau) = -2F - 2C \cos \lambda_0 - \frac{1}{\beta(x_0,y_0)} \big( (2Cy_0)\sin \lambda_0 + (2Cx_0+2C \hat x_0 + e_JD)\cos \lambda_0  \big)^2,$$
$$c(\tau) = 2F - 2C \cos \lambda_0 + \frac{1}{\beta(x_0,y_0)} \big( (2Cx_0+2C \hat x_0 + e_JD)\sin \lambda_0 - (2Cy_0)\cos \lambda_0  \big)^2.$$
%%%%%%%%%%%%%%%%%%%%%%%%%%%%%%%%%%%%%%%%%%%%%%%%%%%%%%%%%%%%%%%%%%%%%%%%%%%%%%%%%%%%%%%%%%%%%%%%%%%%%%%%%%%%%%%%%%%%%%%%%%%%%%%%%%%%%
\begin{utv}

\begin{enumerate}

\item Функции $G_k^\rho, \rho \in \{\lambda,\Lambda,x,y\}$, отвечающие за неоднородность в уравнениях (\ref{xy_eq}), равны:

$$G_k^\rho = \sum_{j=2}^{k} \frac{\mathcal{L}_k^{(j)} g^\rho }{j!}, \quad k \ge 2,$$
где $g^\rho$ - компонента правой части, соответствующая $\rho \in \{ \lambda, \Lambda, x, y \}$,
$$\mathcal{L}_k^{(j)} = \sum_{k_1+...+k_j = k} \mathfrak{D}_{k_1} \cdot ... \cdot \mathfrak{D}_{k_j},$$
$$\mathfrak{D}_k = \left( \begin{pmatrix} \lambda_k\\ \Lambda_k \\ x_k \\ y_k \end{pmatrix},\begin{pmatrix} \frac{\partial}{\partial \lambda} |_{\lambda_0,\Lambda_0,x_0,y_0}\\ \frac{\partial}{\partial \Lambda} |_{\lambda_0,\Lambda_0,x_0,y_0} \\ \frac{\partial}{\partial x} |_{\lambda_0,\Lambda_0,x_0,y_0} \\ \frac{\partial}{\partial y} |_{\lambda_0,\Lambda_0,x_0,y_0} \end{pmatrix} \right),$$
$(\cdot,\cdot)$ -- стандартное скалярное произведение в $\mathbb{R}^4$.

\item Введем классы экспоненциально убывающих и растущих функций:

\[
\mathcal{F}_s = \left\{ f \colon \mathbb{R} \to \mathbb{R} \ \Big| \ \exists C > 0: \ \forall x \in \mathbb{R} \Rightarrow \ |f(x)| \leq C e^{-s|x|} \right\}, s > 0
\]

\[
\mathcal{F}_{-s} = \left\{ f \colon \mathbb{R} \to \mathbb{R} \ \Big| \ \exists C > 0: \ \forall x \in \mathbb{R} \Rightarrow \ |f(x)| \geq C e^{s|x|} \right\}, s > 0
\]

Пусть $\lambda_j, \Lambda_j, x_j, y_j \in \mathcal{F}_s \ \forall \ 0 \le j \le k$, а $g^\rho \in C_b^{\infty} (\mathbb{R}^4)$ (где $C_b^{\infty} (\mathbb{R}^4)$ обозначает множество гладких функций таких, что все их производные ограничены).

Тогда $G_{k+1}^\rho \in \mathcal{F}_{2s}$

\item $$G_k^\lambda = 0 \quad \forall k \ge 1$$  
\end{enumerate}

\end{utv}
%%%%%%%%%%%%%%%%%%%%%%%%%%%%%%%%%%%%%%%%%%%%%%%%%%%%%%%%%%%%%%%%%%%%%%%%%%%%%%%%%%%%%%%%%%%%%%%%%%%%%%%%%%%%%%%%%%%%%%%%%%%%%%%%%%%%%
\textbf{Доказательство:}\nopagebreak[4]

\begin{enumerate}

\item Рассмотрим произвольную функцию $g^\rho(\lambda,\Lambda,x,y)$ и подставим в нее представление (\ref{predst}). Раскладывая в многомерный ряд Тейлора получаем:

$$g^\rho(\lambda,\Lambda,x,y) = g^\rho(\lambda_0,\Lambda_0,x_0,y_0) + \sum_{j=1}^{+\infty} \frac{\mathcal{T}^j g^\rho}{j!},$$
где $\mathcal{T}$ - оператор вида:

\begin{multline*}
\mathcal{T} = 
\left(\sum_{k=1}^{+\infty} \lambda_k \varepsilon^k \right) \frac{\partial}{\partial \lambda} \bigg|_{\lambda_0,\Lambda_0,x_0,y_0}+
\left(\sum_{k=1}^{+\infty} \Lambda_k \varepsilon^k \right) \frac{\partial}{\partial \Lambda} \bigg|_{\lambda_0,\Lambda_0,x_0,y_0}+\\
+\left(\sum_{k=1}^{+\infty} x_k \varepsilon^k \right) \frac{\partial}{\partial x} \bigg|_{\lambda_0,\Lambda_0,x_0,y_0}
+\left(\sum_{k=1}^{+\infty} y_k \varepsilon^k \right) \frac{\partial}{\partial y} \bigg|_{\lambda_0,\Lambda_0,x_0,y_0}.
\end{multline*}

Вводя оператор
$$\mathfrak{D}_k = \left( \begin{pmatrix} \lambda_k\\ \Lambda_k \\ x_k \\ y_k \end{pmatrix},\begin{pmatrix} \frac{\partial}{\partial \lambda} |_{\lambda_0,\Lambda_0,x_0,y_0}\\ \frac{\partial}{\partial \Lambda} |_{\lambda_0,\Lambda_0,x_0,y_0} \\ \frac{\partial}{\partial x} |_{\lambda_0,\Lambda_0,x_0,y_0} \\ \frac{\partial}{\partial y} |_{\lambda_0,\Lambda_0,x_0,y_0} \end{pmatrix} \right),$$
получаем:
$$\mathcal{T} = \sum_{k=1}^{+\infty} \varepsilon^k \mathfrak{D}_k.$$

%$(\cdot,\cdot)$ - стандартное скалярное произведение в $\mathbb{R}^4$.

Рассмотрим выражение для $\mathcal{T}^j$ в котором, явно взяв степень, получаем:

$$\mathcal{T}^j = \left( \sum_{k=1}^{+\infty} \varepsilon^k \mathfrak{D}_k \right)^j = \sum_{k=j}^{+\infty} \varepsilon^k \mathcal{L}_k^{(j)},$$
где:
$$\mathcal{L}_k^{(j)} = \sum_{k_1+...+k_j = k} \mathfrak{D}_{k_1} \cdot ... \cdot \mathfrak{D}_{k_j}.$$

Подставляя этот результат в ряд Тейлора и группируя слагаемые по степеням $\varepsilon$ получаем:

$$g^\rho(\lambda,\Lambda,x,y) = g^\rho(\lambda_0,\Lambda_0,x_0,y_0) + \sum_{k=1}^{+\infty} \varepsilon^k \left( \sum_{j=1}^k \frac{\mathcal{L}_k^{(j)} g^\rho}{j!} \right).$$


Разобьем внутреннюю сумму на 2 части, учитывая $\mathcal{L}_k^{(1)} = \mathfrak{D}_k$:

\begin{equation*}
\begin{aligned}
g^\rho(\lambda,\Lambda,x,y) 
    &= g^\rho(\lambda_0,\Lambda_0,x_0,y_0) + \varepsilon \mathfrak{D}_1 g^\rho + \sum_{k=2}^{+\infty} \varepsilon^k \left( \mathcal{L}_k^{(1)}g^\rho + \sum_{j=2}^k \frac{\mathcal{L}_k^{(j)} g^\rho}{j!} \right) = \\ 
    &= g^\rho(\lambda_0,\Lambda_0,x_0,y_0) + \varepsilon \mathfrak{D}_1 g^\rho + \sum_{k=2}^{+\infty} \varepsilon^k \left( \mathfrak{D}_k g^\rho + \underbrace{\sum_{j=2}^k \frac{\mathcal{L}_k^{(j)} g^\rho}{j!}}_{G_k^\rho} \right) \\ 
    %&= g(\lambda_0,\Lambda_0,x_0,y_0) + \underbrace{\sum_{k=1}^{\infty} \varepsilon^k \mathfrak{D}_k g}_{I} 
    %+ \underbrace{\sum_{k=2}^{\infty} \varepsilon^k \left( \sum_{j=2}^{k} \frac{C_k^{(j)} g }{j!} \right)}_{II = G_k^\rho}.
\end{aligned}
\end{equation*}

\begin{itemize}
\item $\mathfrak{D}_k g^\rho$ - содержит $(\lambda_k, \Lambda_k, x_k, y_k)$ и уйдет в оператор однородного уравнения;
\item $G_k^\rho$ - содержит $(\lambda_j, \Lambda_j, x_j, y_j), \quad j \le k-1$ и входит в неднородность в уравнениях (\ref{xy_eq}).
\end{itemize}

Введенные таким образом $G_k^\rho$ определены только для $k \ge 2$, для $k=1$ по определению положим $G_1^\rho = 0$. В итоге получаем:

\begin{equation}
G_k^\rho = \sum_{j=2}^{k} \frac{\mathcal{L}_k^{(j)} g^\rho }{j!}, \quad k \ge 2,
\label{G_def}
\end{equation}
$$G_1^\rho \equiv 0,$$
$$g^\rho(\lambda,\Lambda,x,y) = g^\rho(\lambda_0,\Lambda_0,x_0,y_0) + \sum_{k=1}^{\infty} \varepsilon^k \mathfrak{D}_k g^\rho + \sum_{k=2}^{+\infty} \varepsilon^k G_k^\rho.$$

\item Так как $g^\rho, \rho \in \{\lambda, \Lambda, x, y\}$ и все ее производные ограничены, а сумма (\ref{G_def}) начинается с $j=2$, то $G_{k+1}^\rho$ будет содержать слагаемые из 2 и более множителей (порядка $\le k$) вида $\mathfrak{D}_i \mathfrak{D}_j g^\rho, i+j \leq k$. Учитывая что каждое решение порядка $\le k$ принадлежит классу $\mathcal{F}_s$, то их произведения и, как следствие, $G_{k+1}^\rho$ будут принадлежать классу $\mathcal{F}_{2s}$.

\item Так как $g^\lambda = \alpha \Lambda$ не содержит нелинейные члены, то любые производные второго порядка и старше будут равны 0. В следствие этого $G_k^\lambda = 0 \quad \forall k \ge 1$. $\blacksquare$
\end{enumerate}
%%%%%%%%%%%%%%%%%%%%%%%%%%%%%%%%%%%%%%%%%%%%%%%%%%%%%%%%%%%%%%%%%%%%%%%%%%%%%%%%%%%%%%%%%%%%%%%%%%%%%%%%%%%%%%%%%%%%%%%%%%%%%%%%%%%%%
Уравнения на нулевой порядок в точности совпадают с уравнениями медленной системы при $\varepsilon = 0$. Так как в данном разделе строим формальные решения, параметризующие $W^{s,u}(0)$, то в качестве нулевого приближения выбираем сепаратрисное решение медленной системы:
\begin{equation*}
    \begin{cases}
        \Lambda_0(\tau) = 0 , \\
        
        \lambda_0(\tau) = \lambda_- (x_{sep}(\tau),y_{sep}(\tau)), \\
        
        x_0(\tau) = x_{sep}(\tau) - \hat x_0, \\
        
        y_0(\tau) = y_{sep}(\tau).\\
    \end{cases}
    %\label{fulltint}
\end{equation*}
%%%%%%%%%%%%%%%%%%%%%%%%%%%%%
\begin{utv}
\begin{enumerate}

    \item Уравнения на $(\Lambda_1,\lambda_1,x_1,y_1)$ являются алгебраичекими по переменным $(\Lambda_1,\lambda_1)$:
    
\begin{equation*}
    \begin{cases}
        \Lambda_1(\tau) = \lambda_{0}', \\
        
        \lambda_1(\tau) = \frac1\beta \Big(x_1 \big( (2Cx_0+2C \hat x_0+e_JD) \sin\lambda_0 - 2Cy_0 \cos\lambda_0 \big) + \\
        +y_1 \big( -2Cy_0 \sin\lambda_0 - (2Cx_0+2C \hat x_0+e_JD) \cos\lambda_0 \big) \Big),
    \end{cases}
    %\label{fulltint}
\end{equation*}

и однородными дифференциальным по переменным $(x_1,y_1)$:
\begin{equation}
\frac{d}{d\tau} \begin{pmatrix} x_k \\ y_k \end{pmatrix} = \mathcal{A}(\tau) \begin{pmatrix} x_k \\ y_k \end{pmatrix}.
\label{utv_odnor}
\end{equation}

    \item Одним из решений (\ref{utv_odnor}) будет:
    
\begin{equation}
    \begin{dcases}
        x_1(\tau)=x_1^{s,u}(\tau)=\frac{dx_0}{d \tau}, \\
        y_1(\tau)=y_1^{s,u}(\tau)=\frac{dy_0}{d \tau}. \\
    \end{dcases}
    %\label{fulltint}
\end{equation}

\end{enumerate}
\end{utv}

%%%%%%%%%%%%%%%%%%%%%%%%%%%%%
\textbf{Доказательство:}\nopagebreak[4]

\begin{enumerate}

\item Рассмотрим абстрактную 4 мерную быстро-медленную систему:

\begin{equation*}
    \begin{cases}
        \dot \Lambda(t)=g^\Lambda(\Lambda,\lambda,x,y), \\
        \dot \lambda(t)=g^\lambda(\Lambda,\lambda,x,y), \\
        \dot x(t)=\varepsilon g^x(\Lambda,\lambda,x,y), \\
        \dot y(t)=\varepsilon g^y(\Lambda,\lambda,x,y), \\
    \end{cases}
\end{equation*}

Переходя к медленному времени $\tau = \varepsilon t$ получаем:
\begin{equation*}
    \begin{cases}
        \varepsilon \Lambda'(\tau)=g^\Lambda(\Lambda,\lambda,x,y), \\
        \varepsilon \lambda'(\tau)=g^\lambda(\Lambda,\lambda,x,y), \\
        x'(\tau)=g^x(\Lambda,\lambda,x,y), \\
        y'(\tau)=g^y(\Lambda,\lambda,x,y), \\
    \end{cases}
\end{equation*}

Обозначим $\v X = (\Lambda,\lambda,x,y)$ и будем искать формальное решение в виде:
$\v X = \sum_{k=0}^\infty \varepsilon^k \v X_k.$
Подставляя это представление в систему и приравнивая члены при одинаковых степенях $\varepsilon$ получаем для младших порядков:

\begin{equation}
    \begin{cases}
        0=g^\Lambda(\v X_0), \\
        0=g^\lambda(\v X_0), \\
        x_0'(\tau)=g^x(\v X_0), \\
        y_0'(\tau)=g^y(\v X_0), \\
    \end{cases}
    \label{slow_sys}
\end{equation}

\begin{equation}
    \begin{cases}
        \Lambda_0'(\tau)=(\nabla g^\Lambda(\v X_0),\v X_1), \\
        \lambda_0'(\tau)=(\nabla g^\lambda(\v X_0),\v X_1), \\
        x_1'(\tau)=(\nabla g^x(\v X_0),\v X_1), \\
        y_1'(\tau)=(\nabla g^y(\v X_0),\v X_1), \\
    \end{cases}
    \label{1st_order}
\end{equation}

Заметим, что первые два уравнения в (\ref{1st_order}) являются линейной неоднородной системой алгебраических уравнений по переменным $(\Lambda_1, \lambda_1)$, представимой в виде:

\begin{align*}
\begin{pmatrix}
(\nabla g^\Lambda)_\Lambda & (\nabla g^\Lambda)_\lambda \\
(\nabla g^\lambda)_\Lambda & (\nabla g^\lambda)_\lambda
\end{pmatrix}
\begin{pmatrix} \Lambda_1 \\ \lambda_1 \end{pmatrix}
= A \begin{pmatrix} \Lambda_1 \\ \lambda_1 \end{pmatrix} =
\begin{pmatrix} \Lambda_0' - (\nabla g^\Lambda)_x x_1 - (\nabla g^\Lambda)_y y_1 \\ \lambda_0' - (\nabla g^\lambda)_x x_1 - (\nabla g^\lambda)_y y_1 \end{pmatrix},
\end{align*}

\begin{equation*}
A = \begin{pmatrix}
(\nabla g^\Lambda)_\Lambda & (\nabla g^\Lambda)_\lambda \\
(\nabla g^\lambda)_\Lambda & (\nabla g^\lambda)_\lambda
\end{pmatrix}.
\end{equation*}

Тогда решение этой системы имеет вид:
\begin{align*}
    \begin{pmatrix} \Lambda_1 \\ \lambda_1 \end{pmatrix} = 
    M \begin{pmatrix} x_1 \\ y_1 \end{pmatrix} + \v C,
\end{align*}

\begin{align*}
    M = \frac{1}{\det A} \begin{pmatrix} 
        -(\nabla g^\lambda)_\lambda (\nabla g^\Lambda)_x + (\nabla g^\Lambda)_\lambda (\nabla g^\lambda)_x & -(\nabla g^\lambda)_\lambda (\nabla g^\Lambda)_y + (\nabla g^\Lambda)_\lambda (\nabla g^\lambda)_y \\ 
         (\nabla g^\lambda)_\Lambda (\nabla g^\Lambda)_x - (\nabla g^\Lambda)_\Lambda (\nabla g^\lambda)_x &  (\nabla g^\lambda)_\Lambda (\nabla g^\Lambda)_y - (\nabla g^\Lambda)_\Lambda (\nabla g^\lambda)_y
    \end{pmatrix},
\end{align*}

$$\v C = \frac{1}{\det A} \begin{pmatrix} (\nabla g^\lambda)_\lambda \Lambda_0' - (\nabla g^\Lambda)_\lambda \lambda_0' \\ -(\nabla g^\lambda)_\Lambda \Lambda_0' + (\nabla g^\Lambda)_\Lambda \lambda_0' \end{pmatrix}.$$

Подставляя решение в последние два уравнения в (\ref{1st_order}) и группируя слагаемые получаем:

\begin{equation}
    \begin{cases}
        x_1'(\tau) = 
            \Big( 
                (\nabla g^x)_x + \big( (\nabla g^x)_\Lambda M_{11} + (\nabla g^x)_\lambda M_{21} \big)
            \Big)x_1 +\\
            \quad \quad \quad + \Big(
                (\nabla g^x)_y + \big( (\nabla g^x)_\Lambda M_{12} + (\nabla g^x)_\lambda M_{22} \big)
            \Big)y_1 +\\
            \quad \quad \quad + \left( \begin{pmatrix} (\nabla g^x)_\Lambda \\ (\nabla g^x)_\lambda \end{pmatrix},\v C \right), \\
            \\
        y_1'(\tau)=
            \Big(
                (\nabla g^y)_x + \big( (\nabla g^y)_\Lambda M_{11} + (\nabla g^y)_\lambda M_{21} \big)
            \Big)x_1 +\\
            \quad \quad \quad + \Big(
                (\nabla g^x)_y + \big( (\nabla g^y)_\Lambda M_{12} + (\nabla g^y)_\lambda M_{22} \big)
            \Big)y_1 +\\
            \quad \quad \quad + \left( \begin{pmatrix} (\nabla g^y)_\Lambda \\ (\nabla g^y)_\lambda \end{pmatrix},\v C \right). \\
    \end{cases}
    \label{x1y1_eq}
\end{equation}

Здесь $(\cdot,\cdot)$ -- стандартное скалярное произведение в $\mathbb{R}^n$.

Таким образом, в общем случае, уравнение для первого приближения является неоднородным, однако для рассматриваемой системы (\ref{fulltn}) неоднородность равна нулю. Действительно, заметим:

$$\Lambda_0 = 0 \quad \Rightarrow \quad \v C = \begin{pmatrix} \frac{\lambda_0'}{\alpha} \\ 0 \end{pmatrix}$$

Также заметим, что компоненты правой части $g^x, g^y$ не зависят от $\Lambda$, поэтому векторы, входящие в неоднородность в уравнениях (\ref{x1y1_eq}) имеют вид:

$$\begin{pmatrix} (\nabla g^x)_\Lambda \\ (\nabla g^x)_\lambda \end{pmatrix} = \begin{pmatrix} 0 \\ * \end{pmatrix},$$
$$\begin{pmatrix} (\nabla g^y)_\Lambda \\ (\nabla g^y)_\lambda \end{pmatrix} = \begin{pmatrix} 0 \\ * \end{pmatrix}.$$

Тогда неоднородность в системе (\ref{x1y1_eq}) вырождается:

$$\left( \begin{pmatrix} (\nabla g^x)_\Lambda \\ (\nabla g^x)_\lambda \end{pmatrix},\v C \right) = \left( \begin{pmatrix} (\nabla g^y)_\Lambda \\ (\nabla g^y)_\lambda \end{pmatrix},\v C \right) = 0.$$

Таким образом, в системе (\ref{fulltn}) дифференциальное уравнение на $(x_1,y_1)$ является однородным.

\item 

Рассмотрим уравнения медленной системы (\ref{slow_sys}). Для простоты приведем выкладки только для компоненты $x_0$.

Продифференцируем уравнение на $x_0$ по времени, при этом будем считать, что $\Lambda_0, \lambda_0$ зависят от $x_0,y_0$:
\begin{multline}
x_0'' = (\nabla g^x)_x x_0' + (\nabla g^x)_y y_0' + (\nabla g^x)_\Lambda \Lambda_0' + (\nabla g^x)_\lambda \lambda_0' = \\
= 
\left( (\nabla g^x)_x + (\nabla g^x)_\Lambda\frac{\partial \Lambda_0}{\partial x} + (\nabla g^x)_\lambda\frac{\partial \lambda_0}{\partial x} \right) x_0' +\\
+\left( (\nabla g^x)_y + (\nabla g^x)_\Lambda\frac{\partial \Lambda_0}{\partial y} + (\nabla g^x)_\lambda\frac{\partial \lambda_0}{\partial y} \right) y_0'.
\label{xdd_eq}
\end{multline}

Заметим, что $\Lambda_0(x_0,y_0), \lambda_0(x_0,y_0)$ задаются неявно через уравнения:
\begin{equation}
    \begin{cases}
        0=g^\Lambda(\v X_0), \\
        0=g^\lambda(\v X_0). \\
    \end{cases}
\end{equation}

Дифференцируем оба уравнения по \(x\):
\[
\begin{cases}
(\nabla g^\Lambda)_x + (\nabla g^\Lambda)_y \underbrace{y_x}_0 + (\nabla g^\Lambda)_\Lambda \cdot \Lambda_x + (\nabla g^\Lambda)_\lambda \cdot \lambda_x = 0, \\
(\nabla g^\lambda)_x + (\nabla g^\lambda)_y \underbrace{y_x}_0 + (\nabla g^\lambda)_\Lambda \cdot \Lambda_x + (\nabla g^\lambda)_\lambda \cdot \lambda_x = 0.
\end{cases}
\]

Полученное выражение можно переписать в матричной форме:
\[
\begin{pmatrix}
(\nabla g^\Lambda)_\Lambda & (\nabla g^\Lambda)_\lambda \\
(\nabla g^\lambda)_\Lambda & (\nabla g^\lambda)_\lambda
\end{pmatrix}
\begin{pmatrix}
\Lambda_x \\
\lambda_x
\end{pmatrix}
= -
\begin{pmatrix}
(\nabla g^\Lambda)_x \\
(\nabla g^\lambda)_x
\end{pmatrix}.
\]

Решая это уравнение относительно $(\Lambda_x,\lambda_x)$ получаем:
$$
\Lambda_x = \frac{(\nabla g^\Lambda)_\lambda (\nabla g^\lambda)_x - (\nabla g^\Lambda)_x (\nabla g^\lambda)_\lambda}{(\nabla g^\Lambda)_\Lambda (\nabla g^\lambda)_\lambda - (\nabla g^\Lambda)_\lambda (\nabla g^\lambda)_\Lambda},
$$
$$
\lambda_x = \frac{(\nabla g^\Lambda)_x (\nabla g^\lambda)_\Lambda - (\nabla g^\Lambda)_\Lambda (\nabla g^\lambda)_x}{(\nabla g^\Lambda)_\Lambda (\nabla g^\lambda)_\lambda - (\nabla g^\Lambda)_\lambda (\nabla g^\lambda)_\Lambda}.
$$

Заметим, что данные частные производные в точности совпадают с коэффициентами матрицы $M$:
$$
\frac{\partial \Lambda_0}{\partial x} = M_{11},
$$
$$
\frac{\partial \lambda_0}{\partial x} = M_{21},
$$
Аналогично можно показать следующее:
$$
\frac{\partial \Lambda_0}{\partial y} = M_{12},
$$
$$
\frac{\partial \lambda_0}{\partial y} = M_{22}.
$$

Подставляя данный результат в (\ref{xdd_eq}) получаем:
\begin{multline}
(x_0')' = (\nabla g^x)_x x_0' + (\nabla g^x)_y y_0' + (\nabla g^x)_\Lambda \Lambda_0' + (\nabla g^x)_\lambda \lambda_0' = \\
= 
\left( (\nabla g^x)_x + (\nabla g^x)_\Lambda M_{11} + (\nabla g^x)_\lambda M_{21} \right) x_0' +\\
+\left( (\nabla g^x)_y + (\nabla g^x)_\Lambda M_{12} + (\nabla g^x)_\lambda M_{22} \right) y_0'.
\label{xdd_eq2}
\end{multline}

Учитывая, что в системе (\ref{fulltn}) неоднородность в уравнениях (\ref{x1y1_eq}) вырождается, вид уравнений (\ref{xdd_eq2}) и (\ref{x1y1_eq}) полностью совпадает.

Заметим, что асимптотика сепаратрисы медленной системы при $\tau \to \pm \infty$ имеет вид: 
\begin{equation*}
    \begin{dcases}
        x_0(\tau) = \frac{\text{const}}{\cosh(f_\pm \cdot \tau)} + \mathcal{O}(e^{-2 f_\pm |\tau|}), \quad \tau \to \pm \infty,\\
        y_0(\tau) = \frac{\text{const} \cdot \sinh(f_\pm \cdot \tau)}{\cosh^2(f_\pm \cdot \tau)} + \mathcal{O}(e^{-2 f_\pm |\tau|}), \quad \tau \to \pm \infty,
    \end{dcases}
\end{equation*}
где $f_\pm = \lim_{\tau \to \pm \infty} \frac{d f^\pm(\tau)}{d\tau} = \frac{J^{+} \chi^{+}}{\sigma^{+}}  \lim_{\tau \to \pm \infty} \frac{P^{+}(f^{+}(\tau))}{S^{+}(f^{+}(\tau))}$. Так как $P^\pm$ и $S^\pm$ являются полиномами второй степени от $\cosh f^\pm$, то пределы существуют и  равны:
 
$$f_+ = f_- = \frac{J^{+} \chi^{+}}{\sigma^{+}} \cdot \frac{1}{U_0} >0$$

Тогда получаем:
$$x_0,y_0 \in \mathcal{F}_{s_0}, \quad s_0 = f_+ = f_- > 0$$

Решение для первого приближения имеет аналогичное поведение на обеих бесконечностях:
\begin{equation*}
    \begin{cases}
        x_1(\pm \infty) = 0, \\
        
        y_1(\pm \infty) = 0. \\
    \end{cases}
\end{equation*}
$$x_1, y_1 \in \mathcal{F}_{s_0}$$

Отсюда получаем, что $(x_1,y_1)$ удовлетворяет граничным условиям и для $W^s(0)$, и для $W^u(0)$, поэтому индексы $s,u$ будем для него опускать. 

$\blacksquare$

\end{enumerate}

Обозначим второе решение однородного уравнения (\ref{utv_odnor}) (линейно независимое к первому) как $(\tilde x_1 (\tau), \tilde y_1 (\tau))$. Т.к. след матрицы $\mathcal{A}$ в однородном уравнении равен 0, то вронскиан, по формуле Лиувилля-Остроградского, является константой и, за счет нормировки второго решения, положим:
$$W(\tau) = x_1(\tau) \tilde y_1 (\tau) - y_1 (\tau) \tilde x_1 (\tau) = 1$$
Также потребуем ортогональность $(x_1, y_1)$ и $(\tilde x_1, \tilde y_1)$.

%Рассмотрим пределы на обеих бесконечностях:
%\begin{equation}
%\begin{aligned}
%W \big|_{\tau \rightarrow -\infty} = \underbrace{x_1 \big|_{\tau \rightarrow -\infty}}_{\rightarrow 0} \tilde y_1 \big|_{\tau \rightarrow -\infty} - \underbrace{y_1 \big|_{\tau \rightarrow -\infty}}_{\rightarrow 0} \tilde x_1 \big|_{\tau \rightarrow -\infty} = 1
%\end{aligned}
%\end{equation}

%\begin{equation}
%\begin{aligned}
%W \big|_{\tau \rightarrow +\infty} = \underbrace{x_1 \big|_{\tau \rightarrow +\infty}}_{\rightarrow 0} \tilde y_1 \big|_{\tau \rightarrow +\infty} - \underbrace{y_1 \big|_{\tau \rightarrow +\infty}}_{\rightarrow 0} \tilde x_1 \big|_{\tau \rightarrow +\infty} = 1
%\end{aligned}
%\end{equation}

%Таким образом $(\tilde x_1, \tilde y_1)$ обязаны экспоненциально расти чтобы компенсировать экспоненциальное убывание $(x_1,y_1)$ на обеих бесконечностях:
%$$\tilde x_1 (\tau), \tilde y_1 (\tau) \in \mathcal{F}_{-s_0}$$

Обозначим:
$$\v u_1(\tau) = \begin{pmatrix} x_1(\tau) \\ y_1(\tau) \end{pmatrix}$$
$$\v{ \tilde{u_1}}(\tau) = \begin{pmatrix} \tilde x_1(\tau) \\ \tilde y_1(\tau) \end{pmatrix}$$

Рассмотрим асимптотику матрицы $\mathcal{A}(\tau)$ при больших $|\tau|$:

\begin{equation}
\mathcal{A}(\tau) = \underbrace{\begin{pmatrix} 
0 & -2F-2C + \frac{(2C \hat x_0 + e_J D)^2}{U_0}\\
2F-2C & 0
\end{pmatrix}}_{\mathcal{A}_0} + \mathcal{O}(e^{-s_0 |\tau|}) \equiv 
\begin{pmatrix} 
0 & b_0\\
c_0 & 0
\end{pmatrix} + \mathcal{O}(e^{-s_0 |\tau|})
\end{equation}

Учитывая значения констант, получаем, что собственные значения $\pm \sqrt{b_0 c_0}$ матрицы $\mathcal{A}_0$ чисто вещественные. Также, учитывая явный вид $b_0,c_0$, заметим, что:
$$\sqrt{b_0 c_0} = s_0.$$

Тогда для положительных $\tau$:

$$\v u_1(\tau) = \text{const} \begin{pmatrix} b_0 \sqrt{c_0} \\ c_0 \sqrt{b_0} \end{pmatrix} e^{-s_0 \tau} + \mathcal{O}(e^{-2s_0 \tau}),$$

$$\v{ \tilde{u_1}}(\tau) = \text{const} \begin{pmatrix} b_0 \sqrt{c_0} \\ -c_0 \sqrt{b_0} \end{pmatrix} e^{s_0 \tau} + \mathcal{O}(1).$$

Аналогично для отрицательных $\tau$:
$$\v u_1(\tau) = \text{const} \begin{pmatrix} b_0 \sqrt{c_0} \\ -c_0 \sqrt{b_0} \end{pmatrix} e^{s_0 \tau} + \mathcal{O}(e^{2s_0 \tau}),$$

$$\v{ \tilde{u_1}}(\tau) = \text{const} \begin{pmatrix} b_0 \sqrt{c_0} \\ c_0 \sqrt{b_0} \end{pmatrix} e^{-s_0 \tau} + \mathcal{O}(1).$$

Таким образом:
$$\v{ \tilde{u_1}}(\tau) \in \mathcal{F}_{-s_0}.$$

Уравнения для приближений старших порядков являются неоднородными дифференциальными уравнениями с той же матрицей $\mathcal{A}$, что и в однородом уравнении на первое приближение. Тогда, с учетом граничных условий, по известным решениям однородного уравнения можно строить решения неоднородного:
\begin{equation}
    \begin{cases}
        \v u_k^u = 
        \v u_1(\tau)\bigint_{\text{ } \tau^u_k}^\tau \left( \tilde y_1(\tau) \tilde G_k^x(\tau) - \tilde x_1(\tau) \tilde G_k^y(\tau) \right) d \tau + 
        \v{ \tilde{u_1}}(\tau) \bigint_{-\infty}^\tau \left( x_1(\tau) \tilde G_k^y(\tau) - y_1(\tau) \tilde G_k^x(\tau) \right) d \tau,\\
        \\
        
        \v u_k^s = 
        \v u_1(\tau)\bigint_{\text{ } \tau^s_k}^{\tau} \left( \tilde y_1(\tau) \tilde G_k^x(\tau) - \tilde x_1(\tau) \tilde G_k^y(\tau) \right) d \tau - 
        \v{ \tilde{u_1}}(\tau) \bigint_{\text{ } \tau}^{+\infty} \left( x_1(\tau) \tilde G_k^y(\tau) - y_1(\tau) \tilde G_k^x(\tau) \right) d \tau. \\
    \end{cases}
    \label{neodnor}
\end{equation}

В силу \textit{Утверждения 2}, подынтегральные выражения во всех интегралах в (\ref{neodnor}) экспоненциально убывают при $\tau \rightarrow \pm \infty$. Следовательно, все интегралы в (\ref{neodnor}) являются ограниченными функциями по $\tau$ при любых нижних пределах интегрирования.

Поскольку $\mathbf{u}_1(\tau)$ убывает при $\tau \rightarrow \pm \infty$, то $\mathbf{u}_k^u$ и $\mathbf{u}_k^s$ будут удовлетворять граничным условиям для любых $\tau^u_k, \tau^s_k \in \mathbb{R}$.

Функция $\tilde{\mathbf{u}_1}(\tau)$ растет при $\tau \rightarrow \pm \infty$. Поэтому в $\mathbf{u}_k^u$ и $\mathbf{u}_k^s$ во втором слагаемом пределы интегрирования выбираются, как указано в (\ref{neodnor}), чтобы компенсировать этот рост на соответствующих бесконечностях. Так как $(x_1 \tilde G_k^y - y_1 \tilde G_k^x) \in \mathcal{F}_{3s_0}$, можно гарантировать, что убывание происходит быстрее, чем рост $\tilde{\mathbf{u}_1}$.

%%%%%%%%%%%%%%%%%%%%%%%%%%%%%%%%%%%%%%%%%%%%%%%%%%%%%%%%%%%%%%%%%%%%%%
\begin{utv}
Пусть по описанной выше процедуре построено формальное решение:
\begin{equation}
    \begin{cases}
\Lambda^s(\tau) = \sum_{k=0}^\infty \varepsilon^k \Lambda_k(\tau)\\
\lambda^s(\tau) = \sum_{k=0}^\infty \varepsilon^k \lambda_k(\tau)\\
x^s(\tau) =       \sum_{k=0}^\infty \varepsilon^k x_k(\tau)\\
y^s(\tau) =       \sum_{k=0}^\infty \varepsilon^k y_k(\tau)\\
\end{cases}
\label{proizv_predst}
\end{equation}
Причем в качестве $(x_1,y_1)$ выбрано следующее решение (\ref{xy_eq}):
\begin{align*}
    \begin{pmatrix} x_1(\tau) \\ y_1(\tau) \end{pmatrix} = \text{const} \cdot \v u_1.
\end{align*}
Константы $\tau_k^{s}$ произвольны.

Тогда будет существовать решение системы (\ref{fulltn}), принадлежащее $W^s(0)$ и, в силу одномерности устойчивого многообразия, однозначно параметризующее его.
Причем асимптотика такого решения будет совпадать с (\ref{proizv_predst}).

Аналогично будет существовать решение системы (\ref{fulltn}), принадлежащее $W^u(0)$, параметризущее его и имеющее аналогичную асимптотику с другими граничными условиями.

\end{utv}
%%%%%%%%%%%%%%%%%%%%%%%%%%%%%%%%%%%%%%%%%%%%%%%%%%%%%%%%%%%%%%%%%%%%%%%

Для исследования расщепления сепаратрис рассмотрим разность устойчивого и неустойчивого решений в каждом порядке:
\begin{multline}
        |\v u_k^u(\tau)-\v u_k^s(\tau)| = \bigg| \v u_1(\tau)\int_{\tau^u_k}^{\tau^s_k} \left( \tilde y_1(\tau) \tilde G_k^x(\tau) - \tilde x_1(\tau) \tilde G_k^y(\tau) \right) d \tau + \\
        +\v{ \tilde{u_1}}(\tau) \int_{-\infty}^{+\infty} \left( x_1(\tau) \tilde G_k^y(\tau) - y_1(\tau) \tilde G_k^x(\tau) \right) d \tau \bigg|
\label{rasshep}
\end{multline}

Существование нулей у такой величины при некоторых $\tau$ будет означать существование пересечения сепаратрис в соответствующем порядке. В таком случае в силу теоремы о единственности решения и одномерности $W^s(0)$ и $W^u(0)$ $\v u_k^s$ и $\v u_k^u$ обязаны совпадать для произвольных $\tau$.

Отметим, что уравнение $|\v u_k^u(\tau_1)-\v u_k^s(\tau_2)|=0$ можно рассматривать с различными моментами времени $\tau_1, \tau_2$. Однако, поскольку достаточно найти хотя бы одно решение такого уравения, мы положим  для простоты $\tau_1=\tau_2=\tau$.


%%%%%%%%%%%%%%%%%%%%%%%%%%%%%%%%%%%%%%%%%%%%%%%%%%%%%%%%%%%%%%%%%%%%%%%%%%%%%%%%%%%%%%%%%%%%%%%%%%%%%%%%%%%%%%%%%%%%%%%%%%%%%%%%%%%%%
\begin{utv}

Для $G_k^\rho$ верна рекуррентная формула:

\begin{equation*}
G_{k+1}^\rho = \frac12 \sum_{m=1}^k \mathfrak{D}_m \mathfrak{D}_{k+1-m} g^\rho + \sum_{m=1}^k \mathfrak{D}_m \hat G_{k+1-m}^\rho,
\end{equation*}
где $\hat G_{k+1}^\rho$ отличается от $G_{k+1}^\rho$ нормировкой:
$$G_k^\rho = \sum_{j=2}^{k} \frac{\mathcal{L}_k^{(j)} g^\rho }{j!},$$
$$\hat G_k^\rho = \sum_{j=2}^{k} \frac{\mathcal{L}_k^{(j)} g^\rho }{(j+1)!},$$
$$\rho \in {\lambda, \Lambda, x, y}.$$
\end{utv}
%%%%%%%%%%%%%%%%%%%%%%%%%%%%%%%%%%%%%%%%%%%%%%%%%%%%%%%%%%%%%%%%%%%%%%%%%%%%%%%%%%%%%%%%%%%%%%%%%%%%%%%%%%%%%%%%%%%%%%%%%%%%%%%%%%%%%
\textbf{Доказательство:}\nopagebreak[4]
$$\mathcal{L}_{k+1}^{(j)}g^\rho = \sum_{m=1}^{k+2-j} \mathfrak{D}_m \mathcal{L}_{k+1-m}^{(j-1)} g^\rho$$

$$G_{k+1}^\rho = \sum_{j=2}^{k+1} \frac{\mathcal{L}_{k+1}^{(j)} g^\rho }{j!} = \sum_{j=2}^{k+1} \sum_{m=1}^{k+2-j} \frac{\mathcal{L}_{k+1-m}^{(j-1)} g^\rho }{j!} = \sum_{m=1}^k \mathfrak{D}_m \sum_{j=2}^{k+2-m} \frac{\mathcal{L}_{k+1-m}^{(j-1)} g^\rho }{j!} =$$
$$= \sum_{m=1}^k \mathfrak{D}_m \sum_{l=1}^{k+1-m} \frac{\mathcal{L}_{k+1-m}^{(l)} g^\rho }{(l+1)!} = \sum_{m=1}^k \mathfrak{D}_m \left( \sum_{l=2}^{k+1-m} \frac{\mathcal{L}_{k+1-m}^{(l)} g^\rho }{(l+1)!} + \frac12 \mathcal{L}_{k+1-m}^{(1)}g^\rho \right) =$$
$$= \sum_{m=1}^k \mathfrak{D}_m \left( \hat G_{k+1-m}^\rho + \frac12 \mathfrak{D}_{k+1-m}g^\rho \right) \blacksquare$$
%%%%%%%%%%%%%%%%%%%%%%%%%%%%%%%%%%%%%%%%%%%%%%%%%%%%%%%%%%%%%%%%%%%%%%%%%%%%%%%%%%%%%%%%%%%%%%%%%%%%%%%%%%%%%%%%%%%%%%%%%%%%%%%%%%%%%
\begin{dfn}
Будем говорить, что функция $f(\tau)$ имеет четность, если $f(\tau)$ является либо четной, либо нечетной по $\tau$.
\end{dfn}

\begin{utv}
\begin{enumerate}

\item 
Если $\lambda_j,\Lambda_j,x_j,y_j, \quad j \le k-1$ как функции $\tau$ имеют четность, то \newline $G_k^x(\tau),\tilde G_k^x(\tau),G_k^\Lambda(\tau), G_k^y(\tau)$ также имеют четность. При этом $G_k^x(\tau),\tilde G_k^x(\tau),G_k^\Lambda(\tau)$ имеют одинаковую четность, а $G_k^y(\tau), \tilde G_k^y(\tau)$ имеют противоположную к 
$G_k^x(\tau)$ четность. 

\item В частности $G_2^x(\tau), \tilde G_2^x(\tau), G_2^\Lambda$ - нечетные, $G_2^y(\tau), \tilde G_2^y(\tau)$ - четные.
\end{enumerate}
\end{utv}
%%%%%%%%%%%%%%%%%%%%%%%%%%%%%%%%%%%%%%%%%%%%%%%%%%%%%%%%%%%%%%%%%%%%%%%%%%%%%%%%%%%%%%%%%%%%%%%%%%%%%%%%%%%%%%%%%%%%%%%%%%%%%%%%%%%%%
\textbf{Доказательство:}\nopagebreak[4]
\begin{enumerate}
\item Рассмотрим четность правой части и различных ее производных по координатам в подстановке нулевого приближения.
Т.к. $x_0(\tau)$ -- четная, $y_0(\tau), \lambda_0(\tau)$ -- нечетные получаем:
$$g^\Lambda|_{\lambda_0(\tau),\Lambda_0(\tau),x_0(\tau),y_0(\tau)} = U(x_0,y_0)\sin\lambda_0-V(x_0,y_0)\cos\lambda_0 \quad \text{ - нечетная},$$
$$g^x|_{\lambda_0(\tau),\Lambda_0(\tau),x_0(\tau),y_0(\tau)} = -\varepsilon \left( 2Fy_0-\frac{\partial U}{\partial y} \cos \lambda_0 - \frac{\partial V}{\partial y} \sin \lambda_0 \right) \quad \text{ - нечетная},$$
$$g^y|_{\lambda_0(\tau),\Lambda_0(\tau),x_0(\tau),y_0(\tau)} = \varepsilon \left( 2F(x_0+\hat x_0)+e_JG -\frac{\partial U}{\partial x} \cos \lambda_0 - \frac{\partial V}{\partial x} \sin \lambda_0 \right) \quad \text{ - четная}.$$

Заметим:
\begin{itemize}
    \item Каждая производная по чётной переменной ($x_0$) сохраняет чётность.
    \item Каждая производная по нечётной переменной ($y_0$ или $\lambda_0$) меняет чётность на противоположную.
\end{itemize}
Таким образом, обозначая четность как $+1$ для четных функций и  $-1$ -  для нечетных функций, получаем:
$$\frac{\partial^{p+q+m}g^\Lambda}{\partial x^p \partial y^q \partial \lambda^m}\bigg|_{\lambda_0(\tau),\Lambda_0(\tau),x_0(\tau),y_0(\tau)} \quad \Leftrightarrow \quad -(-1)^{q+m},$$
\begin{equation}
\frac{\partial^{p+q+m}g^x}{\partial x^p \partial y^q \partial \lambda^m}\bigg|_{\lambda_0(\tau),\Lambda_0(\tau),x_0(\tau),y_0(\tau)} \quad \Leftrightarrow \quad -(-1)^{q+m}, \label{chetnost}
\end{equation}
$$\frac{\partial^{p+q+m}g^y}{\partial x^p \partial y^q \partial \lambda^m}\bigg|_{\lambda_0(\tau),\Lambda_0(\tau),x_0(\tau),y_0(\tau)} \quad \Leftrightarrow \quad (-1)^{q+m}.$$

Заметим, что $G_k^\Lambda(\tau),G_k^x(\tau),G_k^y(\tau)$ имеют вид:
$$G_k^\Lambda(\tau) = \sum_{p_1,q_1,m_1,...} \frac{\partial^{p+q+m}g^\Lambda}{\partial x^p \partial y^q \partial \lambda^m}\bigg|_{\lambda_0(\tau),\Lambda_0(\tau),x_0(\tau),y_0(\tau)} (x_{p_1}y_{q_1}\lambda_{m_1}...),$$
$$G_k^x(\tau) = \sum_{p_1,q_1,m_1,...} \frac{\partial^{p+q+m}g^x}{\partial x^p \partial y^q \partial \lambda^m}\bigg|_{\lambda_0(\tau),\Lambda_0(\tau),x_0(\tau),y_0(\tau)} (x_{p_1}y_{q_1}\lambda_{m_1}...),$$
$$G_k^y(\tau) = \sum_{p_1,q_1,m_1,...} \frac{\partial^{p+q+m}g^y}{\partial x^p \partial y^q \partial \lambda^m}\bigg|_{\lambda_0(\tau),\Lambda_0(\tau),x_0(\tau),y_0(\tau)} (x_{p_1}y_{q_1}\lambda_{m_1}...),$$
и отличаются только производными, множители в скобках у них одинаковые.

Следовательно, если определена четность множителя в скобках (т.е. определена четность всех решений порядка $\le k-1$), то четность $G_k^\Lambda(\tau),G_k^x(\tau),G_k^y(\tau)$ отличается на четность производных.
Учитывая, что четность производных одинакового порядка для $g^\Lambda$ и $g^x$ совпадает, а для $g^y$ отличается, получаем необходимое утверждение.

Четность $\tilde G_k^x, \tilde G_k^y$ наследуется от $G_k^\Lambda(\tau),G_k^x(\tau),G_k^y(\tau)$.

\item Заметим, что $x_0(\tau), y_1(\tau), \tilde x_1, \lambda_1$ -- четные, а $y_0(\tau), x_1(\tau), \tilde y_1$ -- нечетные.

\begin{equation}
G_2^\rho(\tau) = \sum_{p+q+m=2,\quad p,q,m \ge 0, \quad p,q,m \neq 2} \frac{\partial^{2}g^\rho}{\partial x^p \partial y^q \partial \lambda^m}\bigg|_{\lambda_0(\tau),\Lambda_0(\tau),x_0(\tau),y_0(\tau)} x_1^p y_1^q \lambda_1^m
\label{g_2}
\end{equation}

Кроме того, четность $x_1^p y_1^q \lambda_1^m$ равна $(-1)^p$. Тогда, учитывая (\ref{chetnost}), получаем что четность слагаемых в (\ref{g_2}) равна $-(-1)^{p+q+m}$ для $\rho \in \{ \Lambda, x \}$ и $(-1)^{p+q+m}$ для $\rho = y$. Но так как во всех слагаемых $p+q+m=2$, получаем:
$G_2^x(\tau), G_2^\Lambda$ - нечетные, $G_2^y(\tau)$ - четные.

Четность $\tilde G_2^x, \tilde G_2^y$ наследуется от $G_2^\Lambda(\tau),G_2^x(\tau),G_2^y(\tau)$.
$\blacksquare$
\end{enumerate}
%%%%%%%%%%%%%%%%%%%%%%%%%%%%%%%%%%%%%%%%%%%%%%%%%%%%%%%%%%%%%%%%%%%%%%%%%%%%%%%%%%%%%%%%%%%%%%%%%%%%%%%%%%%%%%%%%%%%%%%%%%%%%%%%%%%%%
\begin{consequence}
В силу нечётности подынтегральной функции $x_1(\tau) \tilde G_2^y(\tau) - y_1(\tau) \tilde G_2^x(\tau)$ относительно переменной $\tau$ выполняется:
$$\int_{-\infty}^{+\infty} \left( x_1(\tau) \tilde G_2^y(\tau) - y_1(\tau) \tilde G_2^x(\tau) \right) d \tau = 0.$$

%В таком случае во втором порядке по $\varepsilon$ может происходить расшепление сепаратрис при $\tau^u \ne \tau^s$, причем оно ограничено и стремится к 0 при $\tau \rightarrow \pm \infty$:

%\begin{equation}
%        |\v u_2^u(\tau)-\v u_2^s(\tau)| = | \v u_1(\tau) | \cdot \left| \int_{\tau^u_2}^{\tau^s_2} \left( \tilde y_1(\tau) \tilde G_2^x(\tau) - \tilde x_1(\tau) \tilde G_2^y(\tau) \right) d \tau \right|
%\label{rasshep_res}
%\end{equation}

%При выборе $\tau^u_2 = \tau^s_2$ устойчивое и неустойчивое решения во втором порядке совпадут:



\end{consequence}
%%%%%%%%%%%%%%%%%%%%%%%%%%%%%%%%%%%%%%%%%%%%%%%%%%%%%%%%%%%%%%%%%%%%%%%%%%%%%%%%%%%%%%%%%%%%%%%%%%%%%%%%%%%%%%%%%%%%%%%%%%%%%%%%%%%%%

Поскольку интеграл в \textit{Следствии 1} равен нулю, а подынтегральное выражение нечётное, то следующие функции являются чётными:
$$\int_{-\infty}^{\tau} \left( x_1(\tau) \tilde G_2^y(\tau) - y_1(\tau) \tilde G_2^x(\tau) \right) d \tau \quad \text{ - четная},$$
$$\int_{\tau}^{+\infty} \left( x_1(\tau) \tilde G_2^y(\tau) - y_1(\tau) \tilde G_2^x(\tau) \right) d \tau \quad \text{ - четная}.$$

Так как в общем случае интегралы
$$\int_{\tau^u_2}^{0} \left( \tilde y_1(\tau) \tilde G_2^x(\tau) - \tilde x_1(\tau) \tilde G_2^y(\tau) \right) d \tau \ne 0,$$
$$\int_{\tau^s_2}^{0} \left( \tilde y_1(\tau) \tilde G_2^x(\tau) - \tilde x_1(\tau) \tilde G_2^y(\tau) \right) d \tau \ne 0,$$
то функции 
\begin{equation*}
\begin{aligned}
\int_{\tau^u_2}^{\tau} \left( \tilde y_1(\tau) \tilde G_2^x(\tau) - \tilde x_1(\tau) \tilde G_2^y(\tau) \right) d \tau = \underbrace{\int_{\tau^u_2}^{0} \left( \tilde y_1 \tilde G_2^x - \tilde x_1 \tilde G_2^y \right) d \tau}_{\text{const}} + \underbrace{\int_{0}^{\tau} \overbrace{ \left( \tilde y_1 \tilde G_2^x - \tilde x_1 \tilde G_2^y \right)}^{\text{четная}} d \tau}_{\text{нечетная}},
%g(\lambda,\Lambda,x,y) &= g(\lambda_0,\Lambda_0,x_0,y_0) + \underbrace{\sum_{k=1}^{\infty} \varepsilon^k \mathfrak{D}_k g}_{I} 
%   + \underbrace{\sum_{k=2}^{\infty} \varepsilon^k \left( \sum_{j=2}^{k} \frac{C_k^{(j)} g }{j!} \right)}_{II}.
\end{aligned}
\end{equation*}

\begin{equation*}
\begin{aligned}
\int_{\tau^s}^{\tau} \left( \tilde y_1(\tau) \tilde G_2^x(\tau) - \tilde x_1(\tau) \tilde G_2^y(\tau) \right) d \tau = \underbrace{\int_{\tau^s_2}^{0} \left( \tilde y_1 \tilde G_2^x - \tilde x_1 \tilde G_2^y \right) d \tau}_{\text{const}} + \underbrace{\int_{\tau}^{0} \overbrace{ \left( \tilde y_1 \tilde G_2^x - \tilde x_1 \tilde G_2^y \right)}^{\text{четная}} d \tau}_{\text{нечетная}}
%g(\lambda,\Lambda,x,y) &= g(\lambda_0,\Lambda_0,x_0,y_0) + \underbrace{\sum_{k=1}^{\infty} \varepsilon^k \mathfrak{D}_k g}_{I} 
%   + \underbrace{\sum_{k=2}^{\infty} \varepsilon^k \left( \sum_{j=2}^{k} \frac{C_k^{(j)} g }{j!} \right)}_{II}.
\end{aligned}.
\end{equation*}
не имеют определенной четности.

Следовательно $\v u_2^u(\tau)$ и $\v u_2^s(\tau)$ также не имеют определенной четности при произвольных $\tau_2^{s,u}$. 

В силу \textit{утверждения 4} любое формальное решение, построенное по формулам (\ref{neodnor}) будет параметризовать $W^s(s)$ или $W^s(s)$, соответственно. Поэтому для простоты положим $\tau_k^s = \tau_k^u = 0 \quad \forall k \ge 2$. 

В таком случае:

\begin{multline*}
\begin{aligned}
        |\v u_2^u(\tau)-\v u_2^s(\tau)| = \bigg| \v u_1(\tau) \underbrace{\int_{0}^{0} \left( \tilde y_1(\tau) \tilde G_2^x(\tau) - \tilde x_1(\tau) \tilde G_2^y(\tau) \right) d \tau}_0 + \\
        +\v{ \tilde{u_1}}(\tau) \underbrace{\int_{-\infty}^{+\infty} \left( x_1(\tau) \tilde G_2^y(\tau) - y_1(\tau) \tilde G_2^x(\tau) \right) d \tau}_0 \bigg| = 0
\end{aligned}
\end{multline*}

\begin{multline*}
\begin{aligned}
        |\v u_k^u(\tau)-\v u_k^s(\tau)| = \bigg| \v u_1(\tau) \underbrace{\int_{0}^{0} \left( \tilde y_1(\tau) \tilde G_k^x(\tau) - \tilde x_1(\tau) \tilde G_k^y(\tau) \right) d \tau}_0 + \\
        +\v{ \tilde{u_1}}(\tau) \int_{-\infty}^{+\infty} \left( x_1(\tau) \tilde G_k^y(\tau) - y_1(\tau) \tilde G_k^x(\tau) \right) d \tau \bigg|, \quad k \ge 3.
\end{aligned}
\end{multline*}
%%%%%%%%%%%%%%%%%%%%%%%%%%%%%%%%%%%%%%%%%%%%%%%%%%%%%%%%%%%%%%%%%%%%%%%%%%%%%%%%%%%%%%%%%%%%%%%%%%%%%%%%%%%%%%%%%%%%%%%%%%%%%%%%%%%%%
