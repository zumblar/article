Во многих областях науки возникают системы, процессы в которых имеют различные временные масштабы. Если эти масштабы сильно отличаются, то переменные, отвечающие им, можно разделить на так называемые "быстрые"\, и "медленные"\,. Для описания таких процессов часто используют системы уравнений вида:
\begin{equation}
\begin{cases}
    \frac{dx}{dt} = f(x,y,\varepsilon) ,\\
    \frac{dy}{dt} = \varepsilon g(x,y,\varepsilon),
\end{cases}
\label{full1}
\end{equation}

$$(x,y) \in \mathbb{R}^n \times \mathbb{R}^m,$$
$$f,g \in C^r, r>0,$$
с малым параметром $\varepsilon$. Переменные $y$ называют медленными переменными, переменные $x$, соответственно, называют быстрыми переменными. Малый параметр 
$\varepsilon$ определяет отношение характерных временных масштабов описываемых процессов. Системы вида (\ref{full1}) называются быстро-медленными системами.
Заметим, что вводя новое "медленное"\, время $\tau = \varepsilon t$ система (\ref{full1}) принимает вид:
\begin{equation}
\begin{cases}
    \varepsilon \frac{dx}{d \tau} = f(x,y,\varepsilon), \\
    \frac{dy}{d \tau} = g(x,y,\varepsilon).
\end{cases}
\label{full2}
\end{equation}

При $\varepsilon>0$ обе системы эквивалентны, однако, при $\varepsilon = 0$ получаем так называемые "быструю"\,
\begin{equation}
\begin{cases}
    \frac{dx}{dt} = f(x,y,0), \\
    \frac{dy}{dt} = 0,
\end{cases}
\label{fast}
\end{equation}
и "медленную"\,
\begin{equation}
\begin{cases}
    0 = f(x,y,0), \\
    \frac{dy}{d \tau} = g(x,y,0),
\end{cases}
\label{slow}
\end{equation}
системы, которые не являются эквивалентными.

Очевидно, что динамика медленных переменных в быстрой системе тривиальна, и они являются параметрами для системы уравнений $\frac{dx}{dt} = f(x,y,0)$. При малых $\varepsilon$ система (\ref{fast}) хорошо описывает динамику полной системы с точностью до $O(\varepsilon)$, но только на временах порядка $O(1)$.

С другой стороны, система (\ref{slow}) хорошо описывает поведение полной системы (\ref{full2}) только на множестве нулей функции $f(x,y,0)$, называемом медленным многообразием:
$$M_0 = \{ (x,y): f(x,y,0) = 0 \}.$$

Выразив быстрые переменные на многообразии $M_{0}$ через медленные
$$x = \gamma(y),$$
и подставив в (\ref{slow}), медленная система принимает вид:
$$\frac{dy}{d \tau} = g(\gamma(y), y, 0).$$