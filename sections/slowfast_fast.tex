Устремив в (\ref{fullt}) $\varepsilon$ к нулю, получаем так называемую "быструю"\, систему:

\begin{equation}
    \left\{
    \begin{aligned}
        \dot \Lambda &= U(x,y) \sin \lambda - V(x,y) \cos \lambda,
        &\quad
        \dot \lambda &= \alpha \Lambda, \\[1.5ex]
        \dot x &= 0,
        &\quad
        \dot y &= 0.
    \end{aligned}
    \right.
    \label{fastreal}
\end{equation}

В силу тривиальной динамики переменных $(x, y)$ зависимость функций $U, V$ от медленных переменных будем опускать. Продифференцировав $\lambda$ второй раз по $t$ и подставляя выражение для $\dot \Lambda$, получаем уравнение:
$$\ddot \lambda - \alpha U \sin \lambda + \alpha V \cos \lambda = 0,$$
которое приводится (учитывая $\alpha < 0$) к стандартному уравнению маятника:
$$\ddot \lambda - \alpha \beta \sin \big( \lambda - \lambda_{+}(x,y) \big) = 0, \quad \beta(x,y) \equiv \sqrt{U^2+V^2} > 0.$$

Отметим, что по определению медленное многообразие является множеством неподвижных точек (положений равновесия) "быстрой"\, системы, а именно 
$(\lambda, \Lambda)  = (\lambda_{\pm}(x,y), 0)$. Для определения типа этих положений равновесия разложим правую часть уравнений (\ref{fastreal}) по формуле Тейлора в окрестности положений равновесия:
$$\begin{pmatrix}
  U \sin \lambda - V \cos \lambda \\
  \alpha \Lambda
 \end{pmatrix}
 =
 \begin{pmatrix}
  0 & \alpha \\
  \pm \beta(x,y) & 0
 \end{pmatrix}
 \begin{pmatrix}
  \lambda - \lambda_{\pm}\\
  \Lambda
 \end{pmatrix} + O((\lambda - \lambda_{\pm})^2, \Lambda^2).
$$

Тогда собственные числа линеаризованной системы имеют вид:
\newline
для листа $M_{0,+}$:
$$\zeta = \pm i \sqrt{|\alpha| \beta},$$
\newline
для листа $M_{0,-}$:
$$\zeta = \pm \sqrt{|\alpha| \beta}.$$
В первом случае собственные значения чисто мнимые, а во втором - вещественные. 

Таким образом, лист $M_{0,+}$ состоит из устойчивых положений равновесия быстрой системы,  а лист $M_{0,-}$  - из неустойчивых положений равновесия. Неустойчивое положение равновесия обладает двумерной сепаратрисой, заполненной решениями вида:
\begin{align*}
&\lambda(t, t_0) = \pm 2 \arctan \sinh \big( \alpha \beta (t-t_0) \big) + \lambda_{-}(x,y),\\
&\Lambda(t, t_0) = \frac{\pm 2 \beta}{\cosh \big( \alpha \beta (t-t_0) \big)},
\end{align*}
здесь $t_{0}\in \mathbb{R}$, а выбор знака определяет направление движения по сепаратрисе.