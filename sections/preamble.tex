%\documentclass[a4paper, 12pt]{article}
\usepackage[T2A]{fontenc}
\usepackage[utf8]{inputenc}
%\usepackage[left=1cm,right=1cm,top=1cm,bottom=1.5cm,bindingoffset=0cm]{geometry}
\usepackage{geometry}
\usepackage{setspace}
\usepackage{amsmath}
\usepackage{amssymb}
\usepackage{esint}
\usepackage{mathtools}
\usepackage[english,russian]{babel}
\usepackage{misccorr}                                    %точка после номера секции
\usepackage{graphicx}
%\usepackage{caption}                                     %подпись плавающих             обьектов            
\usepackage{ upgreek }                    
%\captionsetup{labelformat=default,labelsep=period,labelfont=bf}
\usepackage{indentfirst}                                 % Красная строка
\usepackage{float}                                       % можно использовать Н в плавающих обьектах
\usepackage[usenames,dvipsnames,svgnames,table]{xcolor}
\usepackage{lettrine}
\usepackage{fancybox,fancyhdr}
\usepackage{bigints}
\usepackage{verbatim}
\bibliographystyle{unsrt}

\usepackage{amsthm, amssymb}


\geometry{a4paper, left=20mm, right=10mm, top=20mm, bottom=20mm}

%\mathindent=1em 
%\sloppy 
\textwidth = 17 cm 
%\oddsidemargin = -0.5cm 
\topmargin = -2cm \textheight = 25cm 
\setcounter{page}{0}
\DeclareGraphicsExtensions{.pdf,.png,.jpg,.eps}
\bibliographystyle{plain}

\usepackage[T2A]{fontenc} % Поддержка русских букв
\usepackage[utf8]{inputenc} % Кодировка utf8
%\usepackage{pscyr} % Нормальные шрифты

\usepackage{xcolor}
\usepackage{hyperref}

\renewcommand{\v}[1]{\boldsymbol{#1}}

\newtheorem{utv}{Утверждение}
\newtheorem{thm}{Теорема}
\newtheorem{prf}{Доказательство}
\newtheorem{dfn}{Определение}
\newtheorem{consequence}{Следствие}


% Цвета для гиперссылок
\definecolor{linkcolor}{HTML}{4B0082} % цвет ссылок
\definecolor{urlcolor}{HTML}{799B03} % цвет гиперссылок

\hypersetup{pdfstartview=FitH,  linkcolor=linkcolor,urlcolor=urlcolor, colorlinks=true}

%\usepackage[
%backend=biber, 
%sorting=nyt,
%bibstyle=gost-authoryear,
%citestyle=gost-authoryear
%]{biblatex}
%\addbibresource{library}

\setcounter{tocdepth}{4} 

%\renewcommand{\thesection}{(\roman{section})}
%\renewcommand{\thesection}{\Roman{section}}

%\renewcommand{\thesubsection}{(\roman{subsection})}
\renewcommand{\thesubsection}{\arabic{section}.\arabic{subsection}}

% переименовываем  список литературы в "список используемой литературы"
\renewcommand\refname{СПИСОК ЛИТЕРАТУРЫ}
\usepackage{amsthm}

\theoremstyle{plain}
\newtheorem{lemma}{Лемма} 

\theoremstyle{plain}
\newtheorem{theorem}{Теорема} 

\theoremstyle{definition}
\newtheorem{definition}{Условие} 

\theoremstyle{remark}
\newtheorem{remark}{Замечание} 
