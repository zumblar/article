Для доказательства существования решения и его сходимости классически применяется метод мажорант.

\begin{dfn}

Будем говорить, что функция $f(x)$, вещественно-аналитическая в точке 0, мажорируется сходящимся степенным рядом 
$$g(s) = \sum_{n=0}^{+\infty} g_n s^n, \quad s \ge 0 ,$$
$$g_n \ge 0 \quad \forall n \ge 0,$$

если:

\quad Пусть разложение $f(x)$ в ряд Тейлора имеет вид:
$$f(x) = \sum_{n=0}^{+\infty} f_n x^n.$$ 

\quad Тогда
$$\exists C>0: |f_n| \leq C \cdot g_n \quad \forall n \geq 0.$$
Обозначается
$$ f \prec g.$$

%%%%%%%
%Пусть $f(x), g(x)$ - две произвольные функции, вещественно-аналитические в точке 0.
%Пусть их разложение в ряд Тейлора имеет вид:
%$$f(x) = \sum_{n=0}^{+\infty} f_n x^n,$$
%$$g(x) = \sum_{n=0}^{+\infty} g_n x^n.$$ 

%Будем говорить, что функция $g(x)$ мажорирует функцию $f(x)$ если:
%$$\exists C>0: |f_n| \leq C \cdot |g_n| \quad \forall n \geq 0$$
%Обозначается
%$$ f \prec g.$$
\end{dfn}


Введем константу $\varkappa$:
\begin{multline*}
\varkappa = \max \Bigg( 
\sup_{p,q \in \mathbb{R}}\left( \left|\frac{1}{(p-q)i \omega - \xi} \right| \right), 
\sup_{p,q \in \mathbb{R}}\left( \left|\frac{p-q}{(p-q)i \omega - \xi} \right| \right),\\
\sup_{p,q \in \mathbb{R}}\left( \left|\frac{1}{(p-q)i \omega - i\omega} \right| \right), 
\sup_{p,q \in \mathbb{R}}\left( \left|\frac{p-q}{(p-q)i \omega - i\omega} \right| \right) \Bigg) = \\
= \max(\frac{1}{|\xi|}, \frac{1}{|\omega|}) = \frac{1}{|\xi|} = O \left( \frac{1}{\varepsilon} \right) \quad \text{при достаточно малых $\varepsilon$}.
\end{multline*}

Тогда, применяя эту оценку к (\ref{pq_eq}), получаем следующие неравенства для коэффициентов:
\begin{equation}
\begin{cases}
|\{z\}_{p,q}| \le \varkappa |\{\v{\mathcal{G}}_z(z,\eta,a,b)\}_{p,q}| + \varkappa \sum_{r=1}^{+\infty}|{(p-q)\theta_r \{z\}_{p-r,q-r}}|, \\
|\{\eta\}_{p,q}| \le \varkappa |\{\v{\mathcal{G}}_\eta(z,\eta,a,b)\}_{p,q}| + \varkappa \sum_{r=1}^{+\infty}|{(p-q)\theta_r \{\eta\}_{p-r,q-r}}|, \\
|\{a\}_{p,q}| \le \varkappa |\{\v{\mathcal{G}}_a(z,\eta,a,b)\}_{p,q}| + \varkappa \sum_{r=1}^{+\infty}|{(p-q)\theta_r \{a\}_{p-r,q-r}}|, \\
|\{b \}_{p,q}| \le \varkappa |\{\v{\mathcal{G}}_b(z,\eta,a,b)\}_{p,q}| + \varkappa \sum_{r=1}^{+\infty}|{(p-q)\theta_r \{b\}_{p-r,q-r}}|.
\end{cases}
\label{ineq_1}
\end{equation}

В оригинальной работе Зигеля и Мозера \cite{siegel} нелинейная добавка $\v{\mathcal{G}}$ к правой части диагонализованного уравнения мажорируется функцией вида
\begin{align*}
\v{\mathcal{G}}  \prec \frac{1}{1-s},
\end{align*}
где $s = |z|+|\eta|+|a|+|b|$.


Однако такое предположение на $\v{\mathcal{G}}$ дает слишком грубую оценку для радиуса сходимости построенных рядов.

В данном разделе опишем процедуру построения более точной оценки радиуса сходимости.

Учитвая связь старых координат $(\Lambda, \lambda, x, y)$ и новых $(z, \eta, a,b)$ через матрицу $R$, введем новую величину $s$:
%$$s = z + \eta + \varepsilon(a+b)$$
$$s = |z| + |\eta| + \varepsilon(|a|+|b|).$$
Тогда учитывая (\ref{old_var}) получаем:
\begin{equation}
\begin{cases}
\Lambda \prec s,\\
%\Lambda \le s\\
\lambda \prec s,\\
%\lambda \le s\\
x \prec \varepsilon s,\\
%x \le \varepsilon s\\
y \prec \varepsilon s.\\
%y \le \varepsilon s\\
\end{cases}
\label{s_major}
\end{equation}

Используя оценку (\ref{s_major}) и учтывая, что точка $0$ неподвижная, получаем следующую мажоранту для уравнения (\ref{fulltn}):
\begin{equation*}
    \begin{cases}
        - U \sin \lambda + V \cos \lambda \prec (\varepsilon s + \varepsilon^2 s^2)e^{s}, \\
        
        \alpha \Lambda \prec s, \\
        
        -\varepsilon \big( 2Fy-\frac{\partial U}{\partial y} \cos \lambda - \frac{\partial V}{\partial y} \sin \lambda \big) \prec \varepsilon (\varepsilon s + \varepsilon^2 s^2)e^{s}, \\
        
        \varepsilon \big( 2F(x+\hat x_0)+e_JG -\frac{\partial U}{\partial x} \cos \lambda - \frac{\partial V}{\partial x} \sin \lambda \big) \prec \varepsilon (\varepsilon s + \varepsilon^2 s^2)e^{s}. \\
    \end{cases}
\end{equation*}

Домножим слева эту оценку на матрицу $R^{-1}$, тогда имеем:

\begin{equation}
R^{-1} \v F \left(R\begin{pmatrix}
z \\ \eta \\ a \\ b 
\end{pmatrix} \right) \prec
\begin{pmatrix}
(\varepsilon s + \varepsilon^2 s^2)e^{s} + s  \\ 
(\varepsilon s + \varepsilon^2 s^2)e^{s} + s  \\ 
\frac{1}{\varepsilon} (\varepsilon s + \varepsilon^2 s^2)e^{s} + s \\ 
\frac{1}{\varepsilon} (\varepsilon s + \varepsilon^2 s^2)e^{s} + s
\end{pmatrix}.
\label{almoust_g}
\end{equation}

Отбрасывая в (\ref{almoust_g}) линейные члены и мажорируя получившееся выражение, приходим к покомпонентной мажоранте для $\v{\mathcal{G}}$:
\begin{equation}
\begin{cases}
\v{\mathcal{G}}_{z,\eta} &\prec \left( \varepsilon (s + \varepsilon s^2)e^{s} - \varepsilon s \right), \\
\v{\mathcal{G}}_{a,b}    &\prec \left( (s + \varepsilon s^2)e^{s} - s \right).
\label{res_g}
\end{cases}
\end{equation}
Заметим, что слагаемое $-s$ компенсирует линейный член в ряде Тейлора $(s + \varepsilon s^2)e^{s}$ и потому ряд для $(s + \varepsilon s^2)e^{s} - s$ начинается с квадратичного члена.

Для ряда $h$, зависящяго от $\varphi, \psi$, введем обозначение:
$$||h|| = \sum_{p,q} |\{h\}_{p,q}| \varphi^p \psi^q$$.
Также введем обозначения:
\begin{equation*}
\begin{cases}
z^* = z - \varphi,\\
\eta^* = \eta - \psi,\\
a^*=a,\\
b^*=b,\\
\theta^* = \theta - \omega.
\end{cases}
\end{equation*}

Сложим неравенства (\ref{ineq_1}) с весом $\varepsilon$ для $a$ и $b$, домножим на $\varphi^p \psi^q$ и просуммируем. Тогда получаем мажорантное соотношение не содержащее производных:
\begin{equation}
(\varphi+\psi) ||\theta^*|| + S  \prec \varkappa \big( ||\v{\mathcal{G}}_z||+||\v{\mathcal{G}}_\eta||+\varepsilon(||\v{\mathcal{G}}_a||+||\v{\mathcal{G}}_b||) + ||\theta^*||S \big),
\label{major_eq}
\end{equation}
где $S = ||z^*||+||\eta^*||+\varepsilon (||a^*||+||b^*||)$.

Для доказательства сходимости построенных решений достаточно показать, что сходятся ряды $S$ и $||\theta^*||$ в окрестности $\varphi=0, \psi=0$, а так как коэффициенты всех рядов неотрицательны достаточно рассмотреть случай $\varphi=\psi>0$.

Подставим (\ref{res_g}) в (\ref{major_eq}):
\begin{equation*}
2\varphi ||\theta^*|| + S  \prec \varkappa \left( \left(\varepsilon (s + \varepsilon s^2)e^{s} - \varepsilon s \right) + ||\theta^*||S \right).
\label{major_eq2}
\end{equation*}

Для $s$ верна оценка:
$$s \prec ||s|| \prec 2 \varphi + S.$$

Поскольку ряд $S$ начинается с квадратичных членов, введем ряд $Y$ с неотрицательными коэффициентами и начинающийся с линейного члена:
$$Y = 2 ||\theta^*|| + \frac{1}{\varphi} S.$$

Выражая $S$, приходим к
$$S = \varphi Y - 2\varphi ||\theta^*|| \prec \varphi U.$$

Возводем это  выражение в кдвадрат и выразим $||\theta^*|| S$. Тогда
$$||\theta^*|| S = \frac{\varphi^2 Y^2}{4\varphi} - \frac{S^2}{4\varphi} - \frac{4 \varphi^2 ||\theta^*||^2}{4\varphi} \prec \frac{\varphi^2 Y^2}{4\varphi}.$$

Собирая оценки вместе получаем:
\begin{equation}
Y \prec \frac{\varkappa}{4} \left( Y^2 + \varepsilon \left( (2+Y) + \varepsilon \varphi (2+Y)^2 \right) e^{ \varphi (2+Y)} - (2+Y) \right).
\end{equation}

В силу того, что $Y$ строится используя $S$ и $||\theta^*||$, для доказательства сходимости построенных решений достаточно показать, что ряд $Y$ сходится для достаточно малых $\varphi$. Для этого рассмотрим модельное уравнение:

\begin{equation*}
\hat Y = \frac{\varkappa}{4} \left( \hat Y^2 + \varepsilon \left( (2+\hat Y) + \varepsilon \varphi (2+\hat Y)^2 \right) e^{\varphi (2+\hat Y)} - (2+\hat Y) \right),
\end{equation*}

решением которого является неизвестный ряд:
$$\hat Y = \hat Y(\varphi) = \sum_{k=1}^{+\infty} \gamma_k \varphi^k.$$

Исследуя уравнение можно показать что радиус сходимости $R$ этого ряда равен:
$$R =O \left( \frac{1}{\varepsilon \varkappa^2} \right) = \mathcal{O}(\varepsilon).$$
%%%%%%%%%%%%%%%%%%%%%%%%%%%%%%%%%%%%%%%%%%%%%%%%%%%%%%%%%%%%%%%%%%%%%%%%%%%%%%%%%%%%%%%%%%%%%%%%%%%%%%%%%%%%%%%%%%%%%%%%%%%%%%%%%%%%%
