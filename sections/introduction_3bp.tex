В настоящей работе рассматривается плоская ограниченная эллиптическая задача трех тел в случае резонансного отношения периодов 3:1 малого тела и Юпитера. Данная задача является предельным вариантом общей задачи трех тел, исследованию которой посвящено большое число работ. Начиная с исследований Ньютона, Эйлера, Клеро, Даламбера, Лапласа, Лагранжа, Якоби, Коши, Пуанкаре, задача трех тел дала развитие не только анализу, но и многим другим разделам математатики. 

Рассмотрим движение трех тел с гравитационным взаимодействием в пространстве. Будем обозначать величины, связанные с различными телами, индексами $A,J,S$ (условно астероид, Юпитер, Солнце). Тогда уравнения движения принимают вид:
\begin{equation*}
 \begin{cases}
   m_S \v{\ddot{r_S}} = 
   \frac{G m_S m_J}{|\v{r_J-r_S}|^3} (\v{r_J-r_S})+
   \frac{G m_S m_A}{|\v{r_A-r_S}|^3} (\v{r_A-r_S}), 
   \\
   m_J \v{\ddot{r_J}} = 
   \frac{G m_J m_S}{|\v{r_S-r_J}|^3} (\v{r_S-r_J})+
   \frac{G m_J m_A}{|\v{r_A-r_J}|^3} (\v{r_A-r_J}),
   \\
   m_A \v{\ddot{r_A}} = 
   \frac{G m_A m_S}{|\v{r_S-r_A}|^3} (\v{r_S-r_A})+
   \frac{G m_A m_J}{|\v{r_J-r_A}|^3} (\v{r_J-r_A}),
 \end{cases}
\end{equation*}
где $m_{k}$ ($k\in \{A, J, S\}$) - масса $k$-го тела, $r_{k}$ - его радиус-вектор, $G$ - гравитационная постоянная. Поделив уравнения на массы, стоящие в левых частях равенств, можно заметить, что уравнения не теряют смысл и в пределе, когда одна или несколько масс обращаются в нуль.

Данная система является динамической системой с 9 степенями свободы. Однако, если предположить, что движение всех тел происходит в плоскости, то число степеней свободы системы уменьшается до 6, что заметно упрощает дальнейший анализ. При этом задача называется плоской задачей трех тел. Такое предположение является физически обоснованным, поскольку большое число наблюдаемых звездных систем могут считаться плоскими в том смысле, что один из их линейных рамеров много меньше двух других. Кроме того, из астрономических наблюдений известно, что, если астероид влетает в солнечную систему под большим углом, то он не задерживается в ней надолго.

Предположим, что массы тел удовлетворяют соотношению: 
$$m_A \ll m_J \ll m_S.$$ 

В таком случае, учитывая малость $m_A$, слагаемые, содержащие его в правой части, можно отбросить:
\begin{equation*}
 \begin{cases}
   \v{\ddot{r}_{S}} = 
   \frac{G m_J}{|\v{r_J-r_S}|^3} (\v{r_J-r_S}),
   \\
   \v{\ddot{r}_{J}} = 
   \frac{G m_S}{|\v{r_S-r_J}|^3} (\v{r_S-r_J}),
   \\
   \v{\ddot{r}_{A}} = 
   \frac{G m_S}{|\v{r_S-r_A}|^3} (\v{r_S-r_A})+
   \frac{G m_J}{|\v{r_J-r_A}|^3} (\v{r_J-r_A}).
 \end{cases}
\end{equation*}

Заметим, что первые два уравнения не зависят от характеристик астероида и, таким образом, описывают задачу Кеплера о движении двух тел под действием гравитационных сил. Как известно \cite{dub}, задача Кеплера интегрируема. С другой стороны, третье уравнение описывает движение астероида в гравитационном поле двух массивных тел. Такая задача называется ограниченной плоской задачей трех тел.

Перейдем в систему координат Якоби \cite{dub}, связанную с центром масс тел $S$ и $J$:
$$
\v{r} = \v{r_J}-\v{r_S},\quad
\v{\rho} = \frac{m_S}{m_J+m_S} \v{r} - \v{r_J} + \v{r_A},\quad
\v{r_c} = \v{r_J} - \frac{m_S}{m_J+m_S} \v{r}.
$$

Производя замену переменных и учитывая слагаемые, возникающие при переходе в неинерциальную систему отсчета, получаем:
\begin{equation*}
\begin{cases}
\v{\ddot{r_c}} = 0, \\
\v{\ddot{r}} = -\frac{G(m_S+m_J)}{r^3} \v{r},  \\
\v{\ddot{\rho}} = [\v{\rho} \times \frac{d \dot{\Omega}}{dt}]  + 2[\v{\dot{\rho}} \times \v{\Omega}] + [\v{\Omega} \times [\v{\rho} \times \v{\Omega}]] + \frac{1}{m_A} \frac{G m_A m_S}{|\v{r_S-r_A}|^3} (\v{r_S-r_A})+
   \frac{G m_A m_J}{|\v{r_J-r_A}|^3} (\v{r_J-r_A}),  
 \end{cases}
\end{equation*}
где
$$
\v{\Omega} = \frac{[\v{r_c} \times \v{\dot{r_c}}]}{r_c^2}.
$$

Масштабным преобразованием координат и переобозначением констант можно  добиться того, что параметры системы удовлетворяют следующим условиям:
$$
G = 1,\quad 
m_j = \nu,\quad 
m_s = 1 - \nu,\quad
|\v{\Omega}| = 1.
$$

Тогда уравнения движения, описывающие движение малого тела, имеют вид:
\begin{equation*}
 \begin{cases}
   \ddot{\rho_x}=+2 \dot{\rho_y} + \rho_x - \frac{\partial U}{\partial \rho_x}, \\
   \ddot{\rho_y}=-2 \dot{\rho_x} + \rho_y - \frac{\partial U}{\partial \rho_y},
 \end{cases}
\end{equation*}
$$U(r, \v{\rho}) = - \frac{1-\nu}{\sqrt{\big(\rho_x + \nu r \big)^2 + \rho_y^2}} - \frac{\nu}{\sqrt{\big(\rho_x - (1-  \nu ) r \big)^2 + \rho_y^2}}.$$
Полученная система уравнений является системой Лагранжа с лагранжианом:
$$L = \frac{\dot{\rho_x}^2 + \dot{\rho_y}^2}{2} + \frac{{\rho_x}^2 + {\rho_y}^2}{2} + \rho_x \dot{\rho_y} - \rho_y \dot{\rho_x} - U(r,\v{\rho}) .$$
Применяя к лагранжиану преобразование Лежандра и вводя переменные:
$$
x = \rho_x,\quad
y = \rho_y,\quad
p_x = \dot{x} - y,\quad
p_y = \dot{y} + x,
$$
получаем гамильтониан системы:
$$H = \frac{p_x^2 + p_y^2}{2} + p_xy - p_yx + U(r,\v{\rho}).$$

Потенциал в гамильтониане содержит параметр $r = |\v{r}|$ (расстояние между Солнцем и Юпитером), который определяется из уравнения:
$$\v{\ddot{r}} = -\frac{G(m_S+m_J)}{r^3} \v{r},$$
и в общем случае зависит от времени. Это уравнение является частным случаем задачи Кеплера, для которой известно точное решение. 

Известно, что эксцентриситет Юпитера достаточно мал (порядка $e_{J}\approx 0.05$), и потому его орбита близка к круговой. В пределе $e_{J}\to 0$ расстояние $r$ постоянно и мы получаем так называемую круговую плоскую ограниченную задачу трех тел. Данная задача описывается системой с двумя степенями свободы. Следует отметить, что круговая задача интегрируема, т.к. помимо интеграла энергии, она имеет второй интеграл движения - интеграл Якоби:
$$
\dot \rho_{x}^{2} + \dot \rho_{y}^{2} - \rho_{x}^{2} - \rho_{y}^{2} - 2 U(\rho_{x}, \rho_{y}) = {\rm const}.
$$

Однако, как показывают исследования \cite{fejoz, kaloshin}, учет эксцентриситета Юпитера $e_J$ является необходимым для рассмотрения хаотических процессов в системе. В предположении $e_{J}>0$ задача называется плоской эллиптической ограниченной задачей трех тел. 
