\subsubsection{Основные понятие и теоремы}

Одним из методов анализа поведения траекторий системы вблизи медленного многообразия является  геометрическая теория возмущений Феничеля. 
В этом параграфе мы сформулируем некоторые определения и теоремы теории гиперболических множеств и теории Феничеля.

\begin{dfn}
Говорят, что компактное, $C^r$-гладкое, связное многообразие с краем $M \subset R^n$ инвариантно назад (вперёд) относительно потока $\theta_t (x)$, порождённого системой $\frac{dx}{dt} = F(x)$ ($x \in \mathbb{R}^n$, $F(x)$ - $C^r$-гладкое векторное поле), если для всякого $p \in M$ поток $\theta_t (p) \in M$ при $t \leq 0$ 
($t \geq 0$) и для всякого $p \in \partial M$ поле $-F(p)$  направлено внутрь $M$. 

Говорят, что многообразие $M$ инвариантно относительно потока $\theta_t (x)$, если оно одновременно является локально инвариантным вперёд и назад.
\end{dfn}

\begin{dfn}
Компактное, $C^r$-гладкое, связное многообразие с краем $M \subset \mathbb{R}^n$ локально инвариантно относительно потока $\theta_t (x)$, если для всякого $p \in M \setminus \partial M$ найдется интервал $(t_1,t_2)$, $t_1<0$, $t_2>0$ такой, что для любого $t \in (t_1,t_2)$ верно $\theta_t(p) \in M$.
\end{dfn}

\begin{dfn}
Инвариантное множество $M$ называется гиперболическим, если для всякого 
$x_0 \in M$ существуют $\lambda_{1,2} > 0$ такие, что:
$$T_{x_0}\mathbb{R}^{n} = T_{x_0} M \oplus E_{x_0}^u \oplus E_{x_0}^s ,$$
$$\forall v \in E_{x_0}^s, (T_{x_0} \theta_t)v \in E_{\theta_t(x_0)}^s \Rightarrow ||(T_{x_0} \theta_t)v|| \leq e^{-\lambda_1 t} ||v||,\,\, t \geq 0 ,$$
$$\forall v \in E_{x_0}^u, (T_{x_0} \theta_t)v \in E_{\theta_t(x_0)}^u \Rightarrow ||(T_{x_0} \theta_t)v|| \leq e^{\lambda_2 t} ||v||,\,\, t \leq 0 .$$
\end{dfn}

\begin{thm}
\textbf{(Адамар-Перрон):}

Если $M$ - инвариантное гиперболическое множество, то существуют единственные инвариантные устойчивое и неустойчивое многообразия, определеляемые следующим образом:
$$W^s(M) = \{ x: \text{dist} \big(\theta_t(x), M \big) \rightarrow 0, t \rightarrow +\infty \},$$
$$W^u(M) = \{ x: \text{dist} \big(\theta_t(x), M \big) \rightarrow 0, t \rightarrow -\infty \}.$$
\end{thm}

Сформулируем теперь понятие гиперболичности для локально инвариантных множеств:

\begin{dfn}
Локально инвариантное множество $M$ называется гиперболическим, если для всякого $x_0 \in M$ существуют $\lambda_{1,2} > 0$ такие, что:
$$T_{x_0}\mathbb{R}^{n} = T_{x_0} M \oplus E_{x_0}^u \oplus E_{x_0}^s ,$$
$$\forall v \in E_{x_0}^s, (T_{x_0} \theta_t)v \in E_{\theta_t(x_0)}^s \Rightarrow ||(T_{x_0} \theta_t)v|| \leq e^{-\lambda_1 t} ||v||, 0\le t < t_{2} ,$$
$$\forall v \in E_{x_0}^u, (T_{x_0} \theta_t)v \in E_{\theta_t(x_0)}^u \Rightarrow ||(T_{x_0} \theta_t)v|| \leq e^{\lambda_2 t} ||v||, t_{1} < t \le 0 .$$
\end{dfn}

В этом случае справедлив аналог теоремы Адамара-Перрона \cite{berger}

\begin{thm}

Если $M$ - локально инвариантное гиперболическое множество, то существуют локально инвариантные устойчивое и неустойчивое многообразия, определеляемые следующим образом:
$$W^s(M) = \{ x: \text{dist}(\theta_t(x), M) \le \text{dist}(x, M), 0\le t < t_{2} \},$$
$$W^u(M) = \{ x: \text{dist}(\theta_t(x), M) \le \text{dist}(x, M), t_{1} <  t \le 0 \}.$$
\end{thm}

Можно заметить, что инвариантные (локально инвариантные) устойчивое и неустойчивое многообразия расслаиваются на траектории (конечные сегменты траекторий). Дадим следующее

\begin{dfn}
Пусть $M_1$, $M_2$ - два (локально) инвариантных гиперболических множества, $W^{s,u}(M_{1,2})$ - соответствующие им устойчивое и неустойчивое многообразия. Тогда точка $x_0 \in W^u(M_1) \cap W^s(M_2)$ называется гомоклинической, если $M_1 = M_2$, и гетероклинической если $M_1 \neq M_2$. Соответствующие траектории $\theta_t(x_0) \in W^u(M_1) \cap W^s(M_2)$ называются (локально) двояко-асимптотическими.
\end{dfn}

\begin{dfn}
Гомоклиническая (гетероклиническая) точка $x_0$ называется трансверсальной, если:
$$T_{x_0}\mathbb{R}^{n} = T_{x_0}W^s(M_1) \oplus T_{x_0}W^u(M_2).$$
\end{dfn}

Следующая теорема является важным результатом сингулярной теории возмущений:

\begin{thm}
\textbf{(Феничель, \cite{fenichel}):}

Пусть $M_0$ компактное локально инвариантное гиперболическое множество относительно потока $\theta_{t}^{(0)}$, определяемого векторным полем $F_{0}$. Рассмотрим для $\delta>0$ локальные устойчивое и неустойчивое многообразия $W_{loc}^{(s,u)}(M_0) = W_{loc}^{(s,u)}(M_0)\cap U_{\delta}(M_0)$.
Тогда существует достаточно малое $\varepsilon_{0} > 0$ такое, что для любого векторного поля $F_{\varepsilon}$, удовлетворяющего $\vert F_{\varepsilon} - F_{0}\vert < \varepsilon < \varepsilon_{0}$ существует локально инвариантное множество $M_{\varepsilon}$, лежащее в $\varepsilon$ окрестности $M_0$, локально инвариантные устойчивое и неустойчивое многообразия которого $O(\varepsilon)$ близки к 
$W_{loc}^{s,u}(M_0)$.
\end{thm}

Следствием этой теоремы является следующая
\begin{thm}
\textbf{(Феничель, \cite{jones}):}

Пусть $M_0$ компактное подмногообразие (возможно с границей) медленного многообразия системы (\ref{fullt}), гиперболическое относительно (\ref{fast}).

Тогда для любого достаточно малого $\varepsilon > 0$ существует локально инвариантное многообразие $M_{\varepsilon}$ системы (\ref{fullt}) , лежащее в 
$\varepsilon$ окрестности $M_0$.
\end{thm}

Заметим, что данные теоремы гарантируют существование лишь локально инвариантных множеств, вопрос существования глобально инвариантных множеств является более сложной задачей.

Для невозмущенной системы двояко-асимптотические траектории лежат в пересечении устойчивого и неустойчивого многообразий $W^s(M_0) \cap W^u(M_0)$. То же верно и для возмущенной системы, однако, если множество $M_{0}$ имеет непустую границу, то теория Феничеля гарантирует существование лишь локально инвариантных устойчивого и неустойчивого многообразий. Таким образом, их пересечение $W^s(M_\varepsilon) \cap W^u(M_\varepsilon)$ - это множество траекторий, которые ведут себя как гомоклинические только на некотором ограниченном интервале времени.

В дальнейшем будет показано, что существует множество, остающееся глобально инвариантным при возмущении.

